% This is file JFM2esam.tex
% first release v1.0, 20th October 1996
%       release v1.01, 29th October 1996
%       release v1.1, 25th June 1997
%       release v2.0, 27th July 2004
%       release v3.0, 16th July 2014
%   (based on JFMsampl.tex v1.3 for LaTeX2.09)
% Copyright (C) 1996, 1997, 2014 Cambridge University Press

\documentclass{jfm}
<<<<<<< HEAD
\usepackage{graphicx}  \graphicspath{{../FigureNumeric/}}
=======
\usepackage{graphicx}  \graphicspath{{../Figure/}}
>>>>>>> 261252fba4bf040dc7cc35b34ba8820bbadbae79
\usepackage{epstopdf, epsfig}\epstopdfsetup{ suffix=}
\usepackage[colorlinks = true, citecolor = blue]{hyperref}
\usepackage{amsmath}
\usepackage{wasysym}
\usepackage{algorithm}% http://ctan.org/pkg/algorithms
\usepackage{algpseudocode}
<<<<<<< HEAD
\usepackage{natbib} 
=======

>>>>>>> 261252fba4bf040dc7cc35b34ba8820bbadbae79
%%%%%%%%%%%%%%%%%%%%%%%%%%%%%%
\shorttitle{Dynamic Taylor Cone Revisited}
\shortauthor{C. Zhou and S. M. Troian}

\title{Dynamic Taylor Cone Part II: Numeric Scheme}

\author{ Chengzhe Zhou\aff{1}
 \and S. M. Troian\aff{2}   \corresp{\email{stroian@caltech.edu}}} 

\affiliation{\aff{1}Division of Physics, Mathematics and Astronomy, California Institute of Technology,
Pasadena, CA 91125, USA
\aff{2}Department of Applied Physics and Materials Science, 
%Division of Engineering and Applied Science,
California Institute of Technology,
Pasadena, CA 91125, USA}
%%%%%%%%%%%%%%%%%%%%%%%%%%%%%%
% This is file JFM2esam.tex
% first release v1.0, 20th October 1996
%       release v1.01, 29th October 1996
%       release v1.1, 25th June 1997
%       release v2.0, 27th July 2004
%       release v3.0, 16th July 2014
%   (based on JFMsampl.tex v1.3 for LaTeX2.09)
% Copyright (C) 1996, 1997, 2014 Cambridge University Press

\documentclass{jfm}
\usepackage{graphicx}  \graphicspath{{../Figure/}}
\usepackage{epstopdf, epsfig}\epstopdfsetup{ suffix=}
\usepackage[colorlinks = true, citecolor = blue]{hyperref}
\usepackage{amsmath}
\usepackage{wasysym}
\usepackage{algorithm}% http://ctan.org/pkg/algorithms
\usepackage{algpseudocode}

%%%%%%%%%%%%%%%%%%%%%%%%%%%%%%
\shorttitle{Dynamic Taylor Cone Revisited}
\shortauthor{C. Zhou and S. M. Troian}

\title{Dynamic Taylor Cone Part II: Numeric Scheme}

\author{ Chengzhe Zhou\aff{1}
 \and S. M. Troian\aff{2}   \corresp{\email{stroian@caltech.edu}}} 

\affiliation{\aff{1}Division of Physics, Mathematics and Astronomy, California Institute of Technology,
Pasadena, CA 91125, USA
\aff{2}Department of Applied Physics and Materials Science, 
%Division of Engineering and Applied Science,
California Institute of Technology,
Pasadena, CA 91125, USA}
%%%%%%%%%%%%%%%%%%%%%%%%%%%%%%
% This is file JFM2esam.tex
% first release v1.0, 20th October 1996
%       release v1.01, 29th October 1996
%       release v1.1, 25th June 1997
%       release v2.0, 27th July 2004
%       release v3.0, 16th July 2014
%   (based on JFMsampl.tex v1.3 for LaTeX2.09)
% Copyright (C) 1996, 1997, 2014 Cambridge University Press

\documentclass{jfm}
\usepackage{graphicx}  \graphicspath{{../Figure/}}
\usepackage{epstopdf, epsfig}\epstopdfsetup{ suffix=}
\usepackage[colorlinks = true, citecolor = blue]{hyperref}
\usepackage{amsmath}
\usepackage{wasysym}
\usepackage{algorithm}% http://ctan.org/pkg/algorithms
\usepackage{algpseudocode}

%%%%%%%%%%%%%%%%%%%%%%%%%%%%%%
\shorttitle{Dynamic Taylor Cone Revisited}
\shortauthor{C. Zhou and S. M. Troian}

\title{Dynamic Taylor Cone Part II: Numeric Scheme}

\author{ Chengzhe Zhou\aff{1}
 \and S. M. Troian\aff{2}   \corresp{\email{stroian@caltech.edu}}} 

\affiliation{\aff{1}Division of Physics, Mathematics and Astronomy, California Institute of Technology,
Pasadena, CA 91125, USA
\aff{2}Department of Applied Physics and Materials Science, 
%Division of Engineering and Applied Science,
California Institute of Technology,
Pasadena, CA 91125, USA}
%%%%%%%%%%%%%%%%%%%%%%%%%%%%%%
% This is file JFM2esam.tex
% first release v1.0, 20th October 1996
%       release v1.01, 29th October 1996
%       release v1.1, 25th June 1997
%       release v2.0, 27th July 2004
%       release v3.0, 16th July 2014
%   (based on JFMsampl.tex v1.3 for LaTeX2.09)
% Copyright (C) 1996, 1997, 2014 Cambridge University Press

\documentclass{jfm}
\usepackage{graphicx}  \graphicspath{{../Figure/}}
\usepackage{epstopdf, epsfig}\epstopdfsetup{ suffix=}
\usepackage[colorlinks = true, citecolor = blue]{hyperref}
\usepackage{amsmath}
\usepackage{wasysym}
\usepackage{algorithm}% http://ctan.org/pkg/algorithms
\usepackage{algpseudocode}

%%%%%%%%%%%%%%%%%%%%%%%%%%%%%%
\shorttitle{Dynamic Taylor Cone Revisited}
\shortauthor{C. Zhou and S. M. Troian}

\title{Dynamic Taylor Cone Part II: Numeric Scheme}

\author{ Chengzhe Zhou\aff{1}
 \and S. M. Troian\aff{2}   \corresp{\email{stroian@caltech.edu}}} 

\affiliation{\aff{1}Division of Physics, Mathematics and Astronomy, California Institute of Technology,
Pasadena, CA 91125, USA
\aff{2}Department of Applied Physics and Materials Science, 
%Division of Engineering and Applied Science,
California Institute of Technology,
Pasadena, CA 91125, USA}
%%%%%%%%%%%%%%%%%%%%%%%%%%%%%%
\input{taylorCone.macro}

%%%%%%%%%%%%%%%%%%%%%%%%%%%%%%
\begin{document}
\maketitle
\tableofcontents
\begin{abstract}
Numeric scheme \citep{Taylor64}
\end{abstract}

\begin{keywords}
electrohydrodynamics, free surface flow, surface singularity \comment{(Authors should
not enter keywords on the manuscript)}
\end{keywords}

\section{Quintic spline}
Given $N$ marker points $\{\vec{x}_0,\dots,\vec{x}_N\}$,
we first compute $(N-1)$ chord lengths $\{h_0,\dots,h_{N-1}\}$
by euclidean distance between adjacent marker points,
\begin{equation}
h_j=\|\vec{x}_{j+1}-\vec{x}_{j}\|_2\quad\textrm{for}\quad j=0,\dots,N-1
\end{equation}
Then we  introduce $N$ chord coordinates $\{l_0,\dots,l_N\}$ with $l_0=0$ and
\begin{equation}
l_j=\sum_{k=0}^{j-1}h_k\quad\textrm{for}\quad j=1,\dots,N
\end{equation}
Global spline is a collection of local splines $\{\vec{s}^{(0)},\dots,\vec{s}^{(N-1)}\}$
where each $\vec{s}^{(j)}$ is a fifth degree polynomial defined on the interval $l\in[l_j,l_{j+1}]$,
\begin{equation}
\vec{s}^{(j)}(l)=\vec{x}_j+\sum_{k=1}^{5}\vec{c}^{(j)}_k(l-l_j)^k,
\quad\textrm{for}\quad j=0,\dots,N-1
\end{equation}
Continuity between neighboring splines is satisfied up to the fourth derivative,
\begin{equation}
\left.\dd{^{p}\vec{s}^{(j)}}{l^p}\right|_{l_{j+1}}=
\left.\dd{^{p}\vec{s}^{(j+1)}}{l^p}\right|_{l_{j+1}},
\quad\textrm{for}\quad p=0,\dots,4
\end{equation}
The continuity conditions yield five equations
\begin{equation}\left.
\begin{aligned}
\vec{c}^{(j-1)}_3h_{j-1}^3+\vec{c}^{(j-1)}_4h_{j-1}^4+\vec{c}^{(j-1)}_5h_{j-1}^5
&=\vec{x}_j-\vec{x}_{j-1}-\vec{c}^{(j-1)}_1h_{j-1}-\vec{c}^{(j-1)}_2h_{j-1}^2\\
3\vec{c}^{(j-1)}_3h_{j-1}^2+4\vec{c}^{(j-1)}_4h_{j-1}^3+5\vec{c}^{(j-1)}_5h_{j-1}^4
&=-\vec{c}^{(j-1)}_1-2\vec{c}^{(j-1)}_2h_{j-1}+\vec{c}_1^{(j)}\\
6\vec{c}^{(j-1)}_3h_{j-1}+12\vec{c}^{(j-1)}_4h_{j-1}^2+20\vec{c}^{(j-1)}_5h_{j-1}^3
&=-2\vec{c}^{(j-1)}_2+2\vec{c}_2^{(j)}\\
\vec{c}_3^{(j-1)}+4 h_{j-1} \vec{c}_4^{(j-1)}+10 h_{j-1}^2 \vec{c}_5^{(j-1)}
&=\vec{c}_3^{(j)}\\
\vec{c}_4^{(j-1)}+5 h_{j-1} \vec{c}_5^{(j-1)}
&=\vec{c}_4^{(j)}
\end{aligned}\right\}
\end{equation}
We first solve for $\vec{c}^{(j-1)}_3$, $\vec{c}^{(j-1)}_4$ and $\vec{c}^{(j-1)}_5$,
\begin{equation}\label{eq:spline_c3c4c5}\left.
\begin{aligned}
h^3_{j-1}\vec{c}^{(j-1)}_3&=\\
&-6 h_{j-1} \vec{c}_1^{(j-1)}
-3 h_{j-1}^2 \vec{c}_2^{(j-1)}
-4 h_{j-1} \vec{c}_1^{(j)}
+h_{j-1}^2 \vec{c}_2^{(j)}
+10( \vec{x}_j- \vec{x}_{j-1})\\
h^4_{j-1}\vec{c}^{(j-1)}_4&=\\
&+8 h_{j-1} \vec{c}_1^{(j-1)}
+3 h_{j-1}^2 \vec{c}_2^{(j-1)}
+7 h_{j-1} \vec{c}_1^{(j)}
-2 h_{j-1}^2 \vec{c}_2^{(j)}
-15 (\vec{x}_j- \vec{x}_{j-1})\\
h^5_{j-1}\vec{c}^{(j-1)}_5&=\\
&-3 h_{j-1} \vec{c}_1^{(j-1)}
-\phantom{1}h_{j-1}^2 \vec{c}_2^{(j-1)}
-3 h_{j-1} \vec{c}_1^{(j)}
+\phantom{1}h_{j-1}^2 \vec{c}_2^{(j)}
+\phantom{1}6( \vec{x}_j- \vec{x}_{j-1})
\end{aligned}\right\}
\end{equation}
Define $\lambda=h_j/h_{j-1}$. We have $2(N-1)$ equations for $j=1,\dots, N-1$,
\begin{equation}\label{eq:spline_mid}
\left.\begin{aligned}
10\left[+ \lambda^3 \vec{x}_{j-1}- (1+\lambda^3) \vec{x}_j+\vec{x}_{j+1}\right]=
&&4 h_{j} \lambda^2 \vec{c}_1^{(j-1)}
&&+6 h_{j} (\lambda^2-1) \vec{c}_1^{(j)}
&&-4 h_{j} \vec{c}_1^{(j+1)}\\
&&+h_{j}^2 \lambda \vec{c}_2^{(j-1)}
&&-3 h_{j}^2 (1+\lambda) \vec{c}_2^{(j)}
&&+h_{j}^2 \vec{c}_2^{(j+1)}
\\
15 \left[- \lambda^4 \vec{x}_{j-1}+ (\lambda^4-1) \vec{x}_j+\vec{x}_{j+1}\right]=
&&7 h_{j} \lambda^3 \vec{c}_1^{(j-1)}
&&+8 h_{j} (1+\lambda^3) \vec{c}_1^{(j)}
&&+7 h_{j} \vec{c}_1^{(j+1)}\\
&&+2 h_{j}^2 \lambda^2 \vec{c}_2^{(j-1)}
&&+3h_{j}^2 (1- \lambda^2) \vec{c}_2^{(j)}
&&-2 h_{j}^2 \vec{c}_2^{(j+1)}
\end{aligned}\right\}
\end{equation}
for $2(N+1)$ unknowns,
 $\{\vec{c}_1^{(0)},\dots,\vec{c}_1^{(N)}\}$ and  $\{\vec{c}_2^{(0)},\dots,\vec{c}_2^{(N)}\}$.
The imposed boundary conditions at the beginning and end of the global spline
produce four additional equations for
$\{\vec{c}_1^{(0)}, \vec{c}_2^{(0)},\vec{c}_1^{(N)}, \vec{c}_2^{(N)}\}$.
Note we have implicitly introduced a ghost spline $\vec{s}^{(N)}$
which satisfies all continuity conditions with $\vec{s}^{(N-1)}$.
The purpose of $\vec{s}^{(N)}$ is to deal with boundary conditions at the end of spline.

\par In general there are three types of boundary conditions at each end.
Let $x$, $s$ and  ${c}^{(j)}$ be one of the scalar components of vector $\vec{x}_j$,
spline $\vec{s}^{(j)}$ and coefficient $\vec{c}^{(j)}$.
We first consider boundary conditions at $l=l_0$,
\begin{alignat}{3}
\textrm{Even:}\quad&&0&=\dd{s^{(0)}}{l}=\dd{^3s^{(0)}}{l^3}&&\quad\textrm{at}\quad l=l_0\\
\textrm{Odd:}\quad&&0&=s^{(0)}=\dd{^2s^{(0)}}{l^2}=\dd{^4s^{(0)}}{l^4}&&\quad\textrm{at}\quad l=l_0\\
\textrm{Mix:}\quad&&\alpha&=\dd{s^{(0)}}{l},\quad \beta=\dd{^2s^{(0)}}{l^2}&&\quad\textrm{at}\quad l=l_0
\end{alignat}
which lead to two equations,
\begin{subequations}\label{eq:spline_beginbc}
\begin{alignat}{2}
\left.\begin{aligned}
c_1^{(0)}&=0\\
6 h_{0} c_1^{(0)}
+3 h_{0}^2 c_2^{(0)}
+4 h_{0} c_1^{(1)}
-h_{0}^2 c_2^{(1)}
&=-10 x_{0}+10 x_1
\end{aligned}\right\}&&\quad\textrm{Even}\\
\left.\begin{aligned}
-8 h_{0} c_1^{(0)}
-3 h_{0}^2 c_2^{(0)}
-7 h_{0} c_1^{(1)}
+2 h_{0}^2 c_2^{(1)}
&=+15 x_{0}-15 x_1\\
c_2^{(0)}&=0
\end{aligned}\right\}&&\quad\textrm{Odd}\\
\left.\begin{aligned}
c_1^{(0)}&=\alpha\hphantom{\,15 x_{0}-15 x_1}\\
c_2^{(0)}&=\beta/2
\end{aligned}\right\}&&\quad\textrm{Mix}
\end{alignat}
\end{subequations}
We then consider boundary conditions at $l=l_N$,
\begin{alignat}{3}
\textrm{Even:}\quad&&0&=\dd{s^{(N-1)}}{l}=\dd{^3s^{(N-1)}}{l^3}
&&\quad\textrm{at}\quad l=l_N\\
\textrm{Odd:}\quad&&0&=s^{(N-1)}=\dd{^2s^{(N-1)}}{l^2}=\dd{^4s^{(N-1)}}{l^4}
&&\quad\textrm{at}\quad l=l_N\\
\textrm{Mix:}\quad&&\alpha&=\dd{s^{(N-1)}}{l},\quad \beta=\dd{^2s^{(N-1)}}{l^2}
&&\quad\textrm{at}\quad l=l_N
\end{alignat}
which also lead to two equations,
\begin{subequations}\label{eq:spline_endbc}
\begin{alignat}{2}
\left.\begin{aligned}
c_1^{(N)}&=0\\
-4 h_{N-1} {c}_1^{(N-1)}
-h_{N-1}^2 {c}_2^{(N-1)}
-6 h_{N-1} {c}_1^{(N)}
+3 h_{N-1}^2 {c}_2^{(N)}
&=10 {x}_{N-1}-10 {x}_N
\end{aligned}\right\}&&\quad\textrm{Even}\\
\left.\begin{aligned}
-7 h_{N-1} {c}_1^{(N-1)}
-2 h_{N-1}^2 {c}_2^{(N-1)}
-8 h_{N-1} {c}_1^{(N)}
+3 h_{N-1}^2 {c}_2^{(N)}
&=15 {x}_{N-1}-15 {x}_N\\
c_2^{(N)}&=0
\end{aligned}\right\}&&\quad\textrm{Odd}\\
\left.\begin{aligned}
c_1^{(N)}&=\alpha\hphantom{\!\!\!\!15 {x}_{N-1}-15 {x}_N}\\
c_2^{(N)}&=\beta/2
\end{aligned}\right\}&&\quad\textrm{Mix}
\end{alignat}
\end{subequations}
If we arrange the unknowns into a vector,
\begin{equation}
\big[c_1^{(0)},c_2^{(0)},\dots,c_1^{(j)},c_2^{(j)},\dots,c_1^{(N)},c_2^{(N)}\big]
\end{equation}
equations (\ref{eq:spline_mid}) with one of (\ref{eq:spline_beginbc})
and one of (\ref{eq:spline_endbc}) result in a $2(N+1)$-by-$2(N+1)$ system of linear equations,
which corresponds to a banded diagonal sparse matrix with at most 6 non-zero elements in each row.
The rest coefficients $\vec{c}_3^{(j)}$, $\vec{c}_4^{(j)}$ and $\vec{c}_5^{(j)}$
can be reconstruct using equation (\ref{eq:spline_c3c4c5}).

\par It is convenient to re-parametrize each local spline with a new variable $t\in[0,1]$,
\begin{equation}
\vec{s}^{(j)}(t)=\vec{x}_j+\sum_{k=1}^{5}\vec{c}^{(j)}_k  t ^k,
\quad\textrm{for}\quad j=0,\dots,N-1
\end{equation}
by redefining $\vec{c}^{(j)}_k\to \vec{c}^{(j)}_kh_j^k$.
We can construct Lagrange interpolations along the arc-length of the global spline
by the local arc-length $L_j$ and its local fraction $\xi$,
\begin{equation}
L_j = \int_{0}^1\dot{\vec{s}}^{(j)}\,\mathrm{d}t',\quad
\xi(t) =\frac{1}{L_j} \int_{0}^t\dot{\vec{s}}^{(j)}\,\mathrm{d}t',\quad
\end{equation}
Convention for normal vector $\vec{n}$,
\begin{equation}
\vec{n}=\frac{(-\dot{z},\dot{r})}{\|\dot{\vec{s}}\|}
\end{equation}

\section{Numerical approximation}
\subsection{Gaussian  quadrature}
Standard $m$-point Gauss-Legendre quadrature,
\begin{equation}
\int_0^1 f(t)\,\mathrm{d}t\approx \sum_{k=1}^{m}w_k f(x_k),\quad
\int_a^b f(t)\,\mathrm{d}t\Rightarrow
%\int_a^b =\sum_{k=1}^{m}(b-a)w_k f(a+(b-a)x_k),
\begin{aligned}
x_k&\to a+(b-a)x_k\\ w_k&\to (b-a)w_k
\end{aligned}
\end{equation}
Logarithmic-weighted Gauss quadrature,
\begin{equation}
\int_0^1 f(t) \ln t \,\mathrm{d}t\approx -\sum_{k=1}^{m}w_k^\mathrm{log} f(x_k),\quad
\int_0^1 f(t) \ln (1-t) \,\mathrm{d}t\approx -\sum_{k=1}^{m}w_k^\mathrm{log} f(1-x_k)
\end{equation}
When logarithmic singularity $\tau\in (0,1)$,
\begin{align*}
&\int_0^1 f(t) \ln |t-\tau|\,\mathrm{d}t 
=\int_0^\tau f(t) \ln (\tau - t)\,\mathrm{d}t+\int_\tau^1 f(t) \ln (t-\tau)\,\mathrm{d}t\\
=&\,\tau\int_0^1 f(\tau s) \ln (\tau - \tau s)\,\mathrm{d}s
+(1-\tau) \int_0^1 f(\tau + (1-\tau) s) \ln ( (1-\tau) s)\,\mathrm{d}s\\
=&\,\tau\int_0^1 f(\tau s) \ln (1 - s)\,\mathrm{d} s
+(1-\tau) \int_0^1 f(\tau + (1-\tau) s) \ln s\,\mathrm{d}s\\
&\quad+\tau\ln\tau\int_0^1 f(\tau s) \,\mathrm{d} s
+(1-\tau)\ln (1-\tau) \int_0^1 f(\tau + (1-\tau) s) \,\mathrm{d}s
\end{align*}
Logarithmic-weighted Gauss quadrature for singularity $\tau\in (0,1)$,
\begin{equation}\int_0^1 f(t) \ln |t-\tau|\,\mathrm{d}t 
\approx\left\{ 
\begin{aligned}
\tau\ln\tau &\sum_{k=1}^m w_k f\left[\tau x_k\right] \\
+ (1- \tau)\ln(1-\tau) &\sum_{k=1}^m w_k f\left[\tau + (1-\tau)x_k\right]\\
- \tau &\sum_{k=1}^m w_k^\mathrm{log} f\left[\tau(1 -x_k^\mathrm{log})\right]\\
- (1- \tau) &\sum_{k=1}^m w_k^\mathrm{log} f\left[\tau + (1-\tau)x_k^\mathrm{log}\right]
\end{aligned}\right.
\end{equation}

\subsection{Complete elliptic integral}
The complete elliptic integral of the first and second kind,
\begin{equation}
K(m)=\int_0^{\pi/2}\frac{\mathrm{d}\theta}{\sqrt{1-m \sin^2 \theta}},\quad
E(m)=\int_0^{\pi/2}\sqrt{1-m \sin^2 \theta}\,\mathrm{d}\theta
\end{equation}
We can approximate $K(m)$ and $E(m)$ with 
\begin{equation}
K(m)\approx P_K(m) -\ln(1-m) Q_K(m),\quad
E(m)\approx P_E(m) -\ln(1-m) Q_E(m) 
\end{equation}
where $P_K$, $P_E$, $Q_K$, $Q_E$ are tenth order polynomials.
Associate Legendre polynomial of $1/2$ order can be conveniently expressed as,
\begin{equation}
P_{1/2}(x)=\frac{2}{\pi}\left\{2 E\left(\frac{1-x}{2}\right)-K\left(\frac{1-x}{2}\right)\right\}
\end{equation}
From the recursion relations,
\begin{equation}
\begin{aligned}
(1+x^2)\dd{P_\ell(x)}{x}&=(\ell+1) xP_\ell(x)-(\ell+1)P_{\ell+1}(x)\\
\dd{K}{m}&=\frac{E(m)}{2m(1-m)}-\frac{K(m)}{2m}\\
\dd{E}{m}&=\frac{1}{2m}\left[E(m)- K(m)\right]
\end{aligned}
\end{equation}
we can bootstrap associate Legendre polynomials of higher orders, $3/2$ , $5/2$ , ...,
The form is again 
\begin{equation}
P_\ell (x)= c_\ell \left\{P_{\ell, E}(x)E\left(\frac{1-x}{2}\right) +   P_{\ell, K}(x)K\left(\frac{1-x}{2}\right)\right\}
\end{equation}
where $c_\ell$ is some constant. $P_{\ell, E}(x)$ and $P_{\ell, E}(x)$ are polynomials of $(\ell - 1/2)$ order
whose coefficients are computed symbolically and given in Table \ref{tab:LegendrePolyCoef}.
\begin{table}
  \begin{center}
\def~{\hphantom{0}}
  \begin{tabular}{c@{\hskip 20pt}l@{\hskip 20pt} r|l}
$\ell$&$c_\ell$  & $P_{\ell, E}(x)$   &   $P_{\ell, K}(x)$  \\[3pt]
${1}/{2}$&${2}/{\pi }$ & $2$ & $-1$\\[2pt]
${3}/{2}$&${1}/{3 \pi }$ & $16 x$ & $-2 (4 x+1)$\\ [2pt]
${5}/{2}$&${1}/{15 \pi }$ & $4 \left(32 x^2-9\right)$ & $-2 \left(32 x^2+8 x-9\right)$\\[2pt]
${7}/{2}$&${1}/{105 \pi }$ & $64 x \left(24 x^2-13\right)$ & $2 \left(-384 x^3-96 x^2+208 x+25\right)$\\[2pt]
${9}/{2}$&${1}/{315 \pi }$ & $8192 x^4-6528 x^2+588$ & $-2 \left(2048 x^4+512 x^3-1632 x^2-264 x+147\right)$
  \end{tabular}
  \caption{Coefficients of associate Legendre polynomials of $\ell = 1/2$ family}
  \label{tab:LegendrePolyCoef}
  \end{center}
\end{table}
\section{Axisymmetric boundary integral}
Axisymmetric Green's function $G(\vec{\eta}';\vec{\eta})$,
\begin{align}
G(\vec{\eta}';\vec{\eta})&=\frac{1}{\pi\sqrt{a+b}}K(m)\\
\pd{G(\vec{\eta}';\vec{\eta})}{\vec{n}}&=
\frac{1}{\pi r\sqrt{a+b}}\bigg\{
\left[\frac{n_r}{2}+\frac{\vec{n}\cdot(\vec{x}'-\vec{x})}{\|\vec{x}'-\vec{x}\|^2}r\right]E(m)
-\frac{n_r}{2}K(m)\bigg\}
\end{align}
Auxiliary variables,
\begin{equation}
a=r'^2+r^2+(z'-z)^2,\quad b = 2 r' r,\quad m =\frac{2 b}{a+b}
\end{equation}
Note $\eta\to\eta'$ implies $m\to 1$.
Axisymmetric boundary integral equation,
\begin{equation}
\frac{\beta(\vec{\eta}')}{2\pi}\phi(\vec{\eta}')=\int_{\gamma}
\Big\{
G(\vec{\eta}';\vec{\eta})\pd{\phi}{\vec{n}}(\vec{\eta})-
\phi(\vec{\eta})\pd{G(\vec{\eta}';\vec{\eta})}{\vec{n}}(\vec{\eta})
\Big\}
 r \,\mathrm{d}\gamma(\vec{\eta})
\end{equation}
Lagrange interpolation along arc-length,
\begin{equation}
\phi(\vec{\eta})\approx\sum_e \sum_k p_{e_k}\mathcal{N}^{(e)}_k(\xi),\quad
\pd{\phi}{\vec{n}}(\vec{\eta})\approx\sum_e \sum_kq_{e_k}\mathcal{N}^{(e)}_k(\xi)
\end{equation}
where $\mathcal{N}_k$ is the Lagrange basis of the local fractional arc-length variable $\xi$,
\begin{equation}
\mathcal{N}_0=(1 - \xi) (1 - 2\xi),\quad
\mathcal{N}_1=4 \xi (1 - \xi) ,\quad
\mathcal{N}_2=\xi (2 \xi- 1)
\end{equation}
\subsection{Single layer integral}
Single-layer integral reduces to a summation of local integrals over local arc $\gamma_e$ of element $e$,
\begin{align}
\nonumber\int_{\gamma}G(\vec{\eta}';\vec{\eta})
\pd{\phi}{\vec{n}}(\vec{\eta}) r \,\mathrm{d}\gamma(\vec{\eta})
&\approx\sum_e \sum_k q_{e_k}
\int_{\gamma_e}G(\vec{\eta}';\vec{\eta})\mathcal{N}^{(e)}_k(\xi) r \,\mathrm{d}\gamma(\vec{\eta})\\
&=\sum_e \sum_k q_{e_k}
\int_0^1\frac{ r J }{\pi \sqrt{a+b}}\mathcal{N}^{(e)}_k(\xi)K(m)\,\mathrm{d}t
\end{align}
When singularity $\vec{\eta}'\in \gamma_e$ is located at $t=\tau \in[0,1]$,
\begin{align*}
\int_0^1\frac{ r J }{\pi \sqrt{a+b}}\mathcal{N}^{(e)}_kK(m)\,\mathrm{d}t
=&\int_0^1\frac{ r J }{\pi \sqrt{a+b}}\mathcal{N}^{(e)}_k
\left[P_K -Q_K\ln\frac{1-m}{(t-\tau)^2}-2 Q_K\ln |t-\tau| \right]\,\mathrm{d}t
\end{align*}
Introduce helper functions,
\begin{align}
f_K^\mathrm{single}(t)&= \frac{ r J }{\pi \sqrt{a+b}} \\
R_{K/E}(m,\tau)&= P_{K/E}(m) -Q_{K/E}(m) \ln\frac{1-m}{(t-\tau)^2}
\end{align}
If we define integrand $I^{(o)}$ for source location $\vec{\eta}'$ outside of $\gamma_e$
and $I^{(i)}$ for $\vec{\eta}'$ interior to $\gamma_e$,
\begin{align}
I^{(o)}(t) &=  f_K^\mathrm{single}(t) K(m)\mathcal{N}^{(e)}_k(\xi)\\
I^{(i)}(t,\tau) &= f_K^\mathrm{single}(t)R_K(m,\tau)\mathcal{N}^{(e)}_k(\xi)\\
I^{(i)}_\mathrm{log}(t) &=-2 f_K^\mathrm{single} (t)Q_K(m) \mathcal{N}^{(e)}_k(\xi)
\end{align}
then the local single-layer integral can be compactly represented
with a regular part and a logarithmic-singular one,
\begin{align}
\vec{\eta}' \not\in \gamma_e:\,\,\textrm{compute}&\int_0^1I^{(o)}(t)\,\mathrm{d}t\\
\vec{\eta}' \textrm{ at } t= 0:\,\,\textrm{compute}&\int_0^1I^{(i)}(t,0)\,\mathrm{d}t+
\int_0^1I^{(i)}_\mathrm{log}(t)\ln t\,\mathrm{d}t\\
\vec{\eta}' \textrm{ at } t= \tau:\,\,\textrm{compute}&\int_0^1I^{(i)}(t,\tau)\,\mathrm{d}t+
\int_0^1I^{(i)}_\mathrm{log}(t)\ln|\tau-t|\,\mathrm{d}t\\
\vec{\eta}' \textrm{ at } t= 1:\,\,\textrm{compute}&\int_0^1I^{(i)}(t,1)\,\mathrm{d}t+
\int_0^1I^{(i)}_\mathrm{log}(t)\ln(1-t)\,\mathrm{d}t
\end{align}


\subsection{Double layer integral}
Similar to singular-layer integral, double-layer integral reduces to a summation of local integrals over local arc $\gamma_e$ of element $e$,
\begin{align}
&\nonumber\int_{\gamma}\pd{G(\vec{\eta}';\vec{\eta})}{\vec{n}}\phi(\vec{\eta}) r \,\mathrm{d}\gamma(\vec{\eta})
\approx\sum_e \sum_k p_{e_k}
\int_{\gamma_e}\pd{G(\vec{\eta}';\vec{\eta})}{\vec{n}}\mathcal{N}^{(e)}_k(\xi) r \,\mathrm{d}\gamma(\vec{\eta})\\
=&\nonumber\,\sum_e \sum_k p_{e_k}
\int_0^1\mathcal{N}^{(e)}_k(\xi)
\frac{J}{\pi \sqrt{a+b}}\bigg\{
\left[\frac{n_r}{2}+\frac{\vec{n}\cdot(\vec{x}'-\vec{x})}{\|\vec{x}'-\vec{x}\|^2}r\right]E(m)
-\frac{n_r}{2}K(m)\bigg\}
\,\mathrm{d}t\\
=&\,\sum_e \sum_k p_{e_k}
\int_0^1
\frac{\mathcal{N}^{(e)}_k(\xi)}{\pi \sqrt{a+b}}\bigg\{
\left[\frac{\dot{r}(z'-z)-\dot{z}(r'-r)}{(r'-r)^2+(z'-z)^2}r
-\frac{\dot{z}}{2}\right]E(m)
+\frac{\dot{z}}{2}K(m)\bigg\}
\,\mathrm{d}t
\end{align}
Technically we don't need to treat integrand involving elliptic integral of the second kind $E(m)$.
However, convergence rate of standard Gauss-Legendre quadrature  depends magnitude of derivatives.
\begin{equation}
f_E^\mathrm{double}=\frac{1}{\pi \sqrt{a+b}}
\left[\frac{\dot{r}(z'-z)-\dot{z}(r'-r)}{a-b}r
-\frac{\dot{z}}{2}\right],\quad
f_K^\mathrm{double}=\frac{1}{\pi \sqrt{a+b}}
\frac{\dot{z}}{2}
\end{equation}
If we define integrand $I^{(o)}$ for source location $\vec{\eta}'$ outside of $\gamma_e$
and $I^{(i)}$ for $\vec{\eta}'$ interior to $\gamma_e$,
\begin{align}
I^{(o)}(t) &= \left[ f_E^\mathrm{double} E(m)+f_K^\mathrm{double} K(m)\right]\mathcal{N}^{(e)}_k(\xi)\\
I^{(i)}(t,\tau) &= \left[f_E^\mathrm{double}R_E(m,\tau)+ f_K^\mathrm{double}R_K(m,\tau)\right]\mathcal{N}^{(e)}_k(\xi)\\
I^{(i)}_\mathrm{log}(t) &=-2 f_E^\mathrm{double}Q_E(m) -2 f_K^\mathrm{double}Q_K(m) 
\end{align}
When $\vec{\eta'}$ is located the symmetry axis where $r'=0$,
we can simplify the elliptic integrals,
\begin{align}
\int_{\gamma}G(\vec{\eta}';\vec{\eta})
\pd{\phi}{\vec{n}}(\vec{\eta}) r \,\mathrm{d}\gamma(\vec{\eta})
&\approx\sum_e \sum_k q_{e_k}
\int_0^1\mathcal{N}^{(e)}_k(\xi)r\frac{   J }{
2 \sqrt{r+(z-z')^2}}\,\mathrm{d}t\\
\int_{\gamma}\pd{G(\vec{\eta}';\vec{\eta})}{\vec{n}}\phi(\vec{\eta}) r \,\mathrm{d}\gamma(\vec{\eta})
&\approx\sum_e \sum_k p_{e_k}
\int_0^1\mathcal{N}^{(e)}_k(\xi)r \frac{  
\dot{z}r+\dot{r}(z-z')
}{
2\big[r^2+(z'-z)^2\big]^{3/2}
}\,\mathrm{d}t
\end{align}
The above integrands have well-defined limit as $\vec{\eta}\to\vec{\eta}'$
assuming $\ddot{z}/\dot{r}$ is finite at $r=0$.

\subsection{Assembly}
\begin{table}
  \begin{center}
\def~{\hphantom{0}}
  \begin{tabular}{rl|r}
Element order & $o$\\
Number of elements & $n$\\
Number of nodes & $ n \times o +1$\\
Nodal index of source point & $ i$\\
Elemental index of receiving point & $ j$\\
Local node index & $k$ & $0\le k \le o$\\
Indices of elements $\ni$ $i$-th node if $i \textrm{ mod }  o =0$
& $ \lfloor i / o\rfloor-1$, $ \lfloor i / o\rfloor$& check $ 0\le$ id $\le n-1$\\
Indices of elements $\ni$  $i$-th node  if $i \textrm{ mod }  o \neq 0$
& $ \lfloor i / o\rfloor-1$& check $ 0\le$ id $\le n-1$\\
Nodal indices of $j$-th element  & $j \times o +  k$ & $0\le k \le o$\\
  \end{tabular}
  \caption{Index involved}
  \label{tab:kd}
  \end{center}
\end{table}

\begin{algorithm}
  \caption{Counting mismatches between two packed strings
    \label{alg:packed-dna-hamming}}
  \begin{algorithmic}[1]
%\algloop{For}
    \Function{Distance}{$x$, $e$}
    \For{$ 0\le i \le N_x - 1$ } \hfill\text{\# We can parallelize this loop}
    \If{$i \textrm{ mod } o = 0$}     
\State    $I_\mathrm{lower}$  =    $I_\mathrm{upper}$  
          \Else
\State              $I_\mathrm{lower} = \lfloor i/o\rfloor - 1$,\quad
              $I_\mathrm{upper} = \lfloor i/o\rfloor$  
          \EndIf
    \For{$ 0\le i \le N_x - 1$ } 
    
    \EndFor
    
    
    \EndFor
    \EndFunction
  \end{algorithmic}
\end{algorithm}

\begin{equation}
(\mat{B}+\mat{D})\vec{p} =\mat{H}\vec{p} = \mat{S}\vec{q}
\end{equation}
Consider two adjacent boundaries $\gamma_0$ and $\gamma_1$.
We have a block matrix equation,
\begin{equation}
\begin{bmatrix}
\mat{H}_{00}&\mat{H}_{01}\\\mat{H}_{10}&\mat{H}_{11}
\end{bmatrix}
\begin{bmatrix}
\vec{p}_0\\\vec{p}_1
\end{bmatrix}
=\begin{bmatrix}
\mat{S}_{00}&\mat{S}_{01}\\\mat{S}_{10}&\mat{S}_{11}
\end{bmatrix}
\begin{bmatrix}
\vec{q}_0\\\vec{q}_1
\end{bmatrix}
\end{equation}
We have two identical rows in $\mat{H}$ and $\mat{S}$ 
and extra equation for continuity of potential,
\begin{equation}\left.\begin{aligned}
\mathrm{row}_{-1}(\mat{H}_{00})\cdot\vec{p}_0
+\mathrm{row}_{-1}(\mat{H}_{01})\cdot\vec{p}_1
&=\mathrm{row}_{-1}(\mat{S}_{00})\cdot\vec{q}_0
+\mathrm{row}_{-1}(\mat{S}_{01})\cdot\vec{q}_1\\
\mathrm{row}_{0}(\mat{H}_{10})\cdot\vec{p}_0
+\mathrm{row}_{0}(\mat{H}_{11})\cdot\vec{p}_1
&=\mathrm{row}_{0}(\mat{S}_{10})\cdot\vec{q}_0
+\mathrm{row}_{0}(\mat{S}_{11})\cdot\vec{q}_1\\
(p_0)_{-1}-(p_1)_{0}&=0
\end{aligned}\right\}\end{equation}
We simply replace one of identical rows with the continuity condition,
\begin{equation}\begin{aligned}
\mathrm{row}_{-1}(\mat{H}_{00}) &= [0, \,\cdots ,0 , 1],&
\mathrm{row}_{-1}(\mat{H}_{01}) &= [-1,0, \,\cdots ,0 ]\\
\mathrm{row}_{-1}(\mat{S}_{00}) &= [0, \,\cdots ,0 , 0],&
\mathrm{row}_{-1}(\mat{S}_{01}) &= [0,0, \,\cdots ,0 ]
\end{aligned}\end{equation}
For mixed-type boundary condition, i.e., Dirichlet and Neumann on different segments of the boundary,
we simply rearrange the block matrices,
\begin{equation}
\begin{bmatrix}
\mat{H}_{00}&-\mat{S}_{01}\\\mat{H}_{10}&-\mat{S}_{11}
\end{bmatrix}
\begin{bmatrix}
\vec{p}_0\\\vec{q}_1
\end{bmatrix}
=\begin{bmatrix}
\mat{S}_{00}&-\mat{H}_{01}\\\mat{S}_{10}&-\mat{H}_{11}
\end{bmatrix}
\begin{bmatrix}
\vec{q}_0\\\vec{p}_1
\end{bmatrix}
\end{equation}
Note the continuity between discrete potential vectors $\vec{p}_0$ and $\vec{p}_1$
is still implied.

\subsection{Benchmark}
We validate our implementation of boundary integral solver for test problems posed 
on a smooth boundary $\gamma$ and on a piecewise smooth boundary $\gamma_0\cup\gamma_1$.
The latter contains a geometric discontinuity of a 90$^\circ$ corner
at the transition point between $\gamma_0$ and $\gamma_1$.
Consider the following parametrisation of the boundaries,
\begin{align}
\gamma_0&=\left\{
2 \Big(1+\frac{1}{4}\cos(8\theta - \pi)\Big)
\begin{pmatrix}
\cos\theta\\
\sin\theta
\end{pmatrix}
\Bigm|  \theta \in [0,\pi/2]\right\}\\
\gamma_1&=
\left\{
\begin{pmatrix}
t\\0
\end{pmatrix}
\Bigm|  \theta \in [0,3/2]\right\}
\end{align}
The general form of axisymmetric harmonic potential can be constructed from Legendre polynomial $P_\ell$ of order $\ell$.
We consider the interior problem for the potential $\phi$,
\begin{equation}\label{eq:test_parametrisation}
\phi = (r^2+z^2)P_2\Big(\frac{z}{\sqrt{r^2+z^2}}\Big),\quad
\pd{\phi}{\vec{n}} = \vec{n}\cdot (-r , 2 z )
\end{equation}
If we prescribe the potential value of $\phi$ on $\gamma_0$ and 
its normal flux $\partial \phi /\partial \vec{n}$ on $\gamma_1$,
our solver is expected to reproduce $\partial \phi /\partial \vec{n}$ on $\gamma_0$
and $\phi$ on $\gamma_1$,
\begin{equation}\label{eq:test_problem}
\textrm{Test problem:}\quad\textrm{given}\left\{
\begin{aligned}
\phi &\textrm{ on }\gamma_0\\
\partial \phi /\partial \vec{n}&\textrm{ on }\gamma_1
\end{aligned}\right.,
\,\,\textrm{find}\left\{
\begin{aligned}
\partial \phi /\partial \vec{n}&\textrm{ on }\gamma_0\\
\phi &\textrm{ on }\gamma_1
\end{aligned}\right.
\end{equation}
In addition for a plane curve given parametrically as $\gamma(t) = (r(t), z(t))$,
the total curvature $2\kappa$ of the surface of revolution obtained by rotating curve $\gamma$
about the $z$-axis is given by,
\begin{equation}
2\kappa= \frac{\dot{r}\ddot{z} - \dot{z}\ddot{r}}{(\dot{r}^2+\dot{z}^2)^{\frac{3}{2}}}
+\frac{1}{r}\frac{\dot{z}}{\sqrt{\dot{r}^2+\dot{z}^2}}
\end{equation}
We verify the accuracy of quintic spline interpolation against the analytic form derived from
the parametrisation (\ref{eq:test_parametrisation}). All errors between numerical and analytical solutions
are measured in the $l^\infty$-norm,
\begin{align}
\|\textrm{err}_\mathrm{d}\|_\infty &= 
\max_{\vec{x}_i\in \gamma_1}{\lvert p^\mathrm{num}_i-\phi(\vec{x}_i)\rvert}\\
\|\textrm{err}_\mathrm{n}\|_\infty &= 
\max_{\vec{x}_i\in \gamma_0}{\lvert q^\mathrm{num}_i-\partial\phi/\partial\vec{n}(\vec{x}_i)\rvert}\\
\|\textrm{err}_\mathrm{c}\|_\infty &= 
\max_{\vec{x}_i\in \gamma_0}{\lvert \kappa^\mathrm{spline}_i-\kappa(\vec{x}_i)\rvert}
\end{align}
In figure \ref{fig:test_result}(a) and \ref{fig:test_result}(b) 
we plot these errors against the total number of degree of freedom (DOF) used by the solver.
\begin{figure}
  \centering
  \includegraphics[height =  5.2cm]{test.pdf}% Images in 100% size
  \caption{
Error convergence  between numerical and analytic solutions measured in $l^\infty$-norm against degrees of freedom (DOF) used for the test problem (\ref{eq:test_problem}):
(a) Boundaries $\gamma_0$ (solid) and $\gamma_1$ (dashed) given by parametrisation (\ref{eq:test_parametrisation}).
(b) Convergence of total curvature ($\blacktriangledown$) on $\gamma_0$;
convergence of Neumann data on $\gamma_0$ with linear ($\CIRCLE $) and quadratic ($\blacksquare$) shape functions;
convergence of Dirichlet data on $\gamma_1$ with linear ($\ocircle $) and quadratic ($\Box$) shape functions;
linear fit of the last five points of each error convergence (inset).
  }
\label{fig:test_result}
\end{figure}

\section{Self-similar Taylor cone}
In the last section we outline the numerical procedure to solve Laplace equation subject to mixed boundary conditions.
Initial guess is a $C^3$ continuous function $f_\mathrm{guess}$,
\begin{equation}
f_\mathrm{guess}(r)=\left\{\begin{aligned}
&f_0r+f_2r^2+f_4r^4+f_6r^6,&&\textrm{if }r\le r_c\\
&c_0 r + \frac{c_1}{\sqrt{r}} +\frac{c_3}{r^{7/2}}+\frac{c_4}{r^{5}},&&\textrm{if }r>r_c
\end{aligned}\right.
\end{equation}
where $r_c$ is the connection point where continuities up to the third derivative are enforced.
Given a reasonable $c_1$, function $f_\mathrm{guess}(r)$ agrees with analytic prediction in the far-field.
The effect of varying $r_c$ and $c_1$ is illustrated in figure \ref{fig:initialGuess}.
\begin{figure}
  \centering
  \includegraphics[height =  4.2cm]{fig02.pdf}% Images in 100% size
  \caption{$C^3$ function $f_\mathrm{guess}(r)$: increasing $r_c$ (right) and $c_1$ (left)  }
\label{fig:initialGuess}
\end{figure}




























\bibliographystyle{jfm}
% Note the spaces between the initials
\bibliography{taylorCone}

\end{document}


%%%%%%%%%%%%%%%%%%%%%%%%%%%%%%
\begin{document}
\maketitle
\tableofcontents
\begin{abstract}
Numeric scheme \citep{Taylor64}
\end{abstract}

\begin{keywords}
electrohydrodynamics, free surface flow, surface singularity \comment{(Authors should
not enter keywords on the manuscript)}
\end{keywords}

\section{Quintic spline}
Given $N$ marker points $\{\vec{x}_0,\dots,\vec{x}_N\}$,
we first compute $(N-1)$ chord lengths $\{h_0,\dots,h_{N-1}\}$
by euclidean distance between adjacent marker points,
\begin{equation}
h_j=\|\vec{x}_{j+1}-\vec{x}_{j}\|_2\quad\textrm{for}\quad j=0,\dots,N-1
\end{equation}
Then we  introduce $N$ chord coordinates $\{l_0,\dots,l_N\}$ with $l_0=0$ and
\begin{equation}
l_j=\sum_{k=0}^{j-1}h_k\quad\textrm{for}\quad j=1,\dots,N
\end{equation}
Global spline is a collection of local splines $\{\vec{s}^{(0)},\dots,\vec{s}^{(N-1)}\}$
where each $\vec{s}^{(j)}$ is a fifth degree polynomial defined on the interval $l\in[l_j,l_{j+1}]$,
\begin{equation}
\vec{s}^{(j)}(l)=\vec{x}_j+\sum_{k=1}^{5}\vec{c}^{(j)}_k(l-l_j)^k,
\quad\textrm{for}\quad j=0,\dots,N-1
\end{equation}
Continuity between neighboring splines is satisfied up to the fourth derivative,
\begin{equation}
\left.\dd{^{p}\vec{s}^{(j)}}{l^p}\right|_{l_{j+1}}=
\left.\dd{^{p}\vec{s}^{(j+1)}}{l^p}\right|_{l_{j+1}},
\quad\textrm{for}\quad p=0,\dots,4
\end{equation}
The continuity conditions yield five equations
\begin{equation}\left.
\begin{aligned}
\vec{c}^{(j-1)}_3h_{j-1}^3+\vec{c}^{(j-1)}_4h_{j-1}^4+\vec{c}^{(j-1)}_5h_{j-1}^5
&=\vec{x}_j-\vec{x}_{j-1}-\vec{c}^{(j-1)}_1h_{j-1}-\vec{c}^{(j-1)}_2h_{j-1}^2\\
3\vec{c}^{(j-1)}_3h_{j-1}^2+4\vec{c}^{(j-1)}_4h_{j-1}^3+5\vec{c}^{(j-1)}_5h_{j-1}^4
&=-\vec{c}^{(j-1)}_1-2\vec{c}^{(j-1)}_2h_{j-1}+\vec{c}_1^{(j)}\\
6\vec{c}^{(j-1)}_3h_{j-1}+12\vec{c}^{(j-1)}_4h_{j-1}^2+20\vec{c}^{(j-1)}_5h_{j-1}^3
&=-2\vec{c}^{(j-1)}_2+2\vec{c}_2^{(j)}\\
\vec{c}_3^{(j-1)}+4 h_{j-1} \vec{c}_4^{(j-1)}+10 h_{j-1}^2 \vec{c}_5^{(j-1)}
&=\vec{c}_3^{(j)}\\
\vec{c}_4^{(j-1)}+5 h_{j-1} \vec{c}_5^{(j-1)}
&=\vec{c}_4^{(j)}
\end{aligned}\right\}
\end{equation}
We first solve for $\vec{c}^{(j-1)}_3$, $\vec{c}^{(j-1)}_4$ and $\vec{c}^{(j-1)}_5$,
\begin{equation}\label{eq:spline_c3c4c5}\left.
\begin{aligned}
h^3_{j-1}\vec{c}^{(j-1)}_3&=\\
&-6 h_{j-1} \vec{c}_1^{(j-1)}
-3 h_{j-1}^2 \vec{c}_2^{(j-1)}
-4 h_{j-1} \vec{c}_1^{(j)}
+h_{j-1}^2 \vec{c}_2^{(j)}
+10( \vec{x}_j- \vec{x}_{j-1})\\
h^4_{j-1}\vec{c}^{(j-1)}_4&=\\
&+8 h_{j-1} \vec{c}_1^{(j-1)}
+3 h_{j-1}^2 \vec{c}_2^{(j-1)}
+7 h_{j-1} \vec{c}_1^{(j)}
-2 h_{j-1}^2 \vec{c}_2^{(j)}
-15 (\vec{x}_j- \vec{x}_{j-1})\\
h^5_{j-1}\vec{c}^{(j-1)}_5&=\\
&-3 h_{j-1} \vec{c}_1^{(j-1)}
-\phantom{1}h_{j-1}^2 \vec{c}_2^{(j-1)}
-3 h_{j-1} \vec{c}_1^{(j)}
+\phantom{1}h_{j-1}^2 \vec{c}_2^{(j)}
+\phantom{1}6( \vec{x}_j- \vec{x}_{j-1})
\end{aligned}\right\}
\end{equation}
Define $\lambda=h_j/h_{j-1}$. We have $2(N-1)$ equations for $j=1,\dots, N-1$,
\begin{equation}\label{eq:spline_mid}
\left.\begin{aligned}
10\left[+ \lambda^3 \vec{x}_{j-1}- (1+\lambda^3) \vec{x}_j+\vec{x}_{j+1}\right]=
&&4 h_{j} \lambda^2 \vec{c}_1^{(j-1)}
&&+6 h_{j} (\lambda^2-1) \vec{c}_1^{(j)}
&&-4 h_{j} \vec{c}_1^{(j+1)}\\
&&+h_{j}^2 \lambda \vec{c}_2^{(j-1)}
&&-3 h_{j}^2 (1+\lambda) \vec{c}_2^{(j)}
&&+h_{j}^2 \vec{c}_2^{(j+1)}
\\
15 \left[- \lambda^4 \vec{x}_{j-1}+ (\lambda^4-1) \vec{x}_j+\vec{x}_{j+1}\right]=
&&7 h_{j} \lambda^3 \vec{c}_1^{(j-1)}
&&+8 h_{j} (1+\lambda^3) \vec{c}_1^{(j)}
&&+7 h_{j} \vec{c}_1^{(j+1)}\\
&&+2 h_{j}^2 \lambda^2 \vec{c}_2^{(j-1)}
&&+3h_{j}^2 (1- \lambda^2) \vec{c}_2^{(j)}
&&-2 h_{j}^2 \vec{c}_2^{(j+1)}
\end{aligned}\right\}
\end{equation}
for $2(N+1)$ unknowns,
 $\{\vec{c}_1^{(0)},\dots,\vec{c}_1^{(N)}\}$ and  $\{\vec{c}_2^{(0)},\dots,\vec{c}_2^{(N)}\}$.
The imposed boundary conditions at the beginning and end of the global spline
produce four additional equations for
$\{\vec{c}_1^{(0)}, \vec{c}_2^{(0)},\vec{c}_1^{(N)}, \vec{c}_2^{(N)}\}$.
Note we have implicitly introduced a ghost spline $\vec{s}^{(N)}$
which satisfies all continuity conditions with $\vec{s}^{(N-1)}$.
The purpose of $\vec{s}^{(N)}$ is to deal with boundary conditions at the end of spline.

\par In general there are three types of boundary conditions at each end.
Let $x$, $s$ and  ${c}^{(j)}$ be one of the scalar components of vector $\vec{x}_j$,
spline $\vec{s}^{(j)}$ and coefficient $\vec{c}^{(j)}$.
We first consider boundary conditions at $l=l_0$,
\begin{alignat}{3}
\textrm{Even:}\quad&&0&=\dd{s^{(0)}}{l}=\dd{^3s^{(0)}}{l^3}&&\quad\textrm{at}\quad l=l_0\\
\textrm{Odd:}\quad&&0&=s^{(0)}=\dd{^2s^{(0)}}{l^2}=\dd{^4s^{(0)}}{l^4}&&\quad\textrm{at}\quad l=l_0\\
\textrm{Mix:}\quad&&\alpha&=\dd{s^{(0)}}{l},\quad \beta=\dd{^2s^{(0)}}{l^2}&&\quad\textrm{at}\quad l=l_0
\end{alignat}
which lead to two equations,
\begin{subequations}\label{eq:spline_beginbc}
\begin{alignat}{2}
\left.\begin{aligned}
c_1^{(0)}&=0\\
6 h_{0} c_1^{(0)}
+3 h_{0}^2 c_2^{(0)}
+4 h_{0} c_1^{(1)}
-h_{0}^2 c_2^{(1)}
&=-10 x_{0}+10 x_1
\end{aligned}\right\}&&\quad\textrm{Even}\\
\left.\begin{aligned}
-8 h_{0} c_1^{(0)}
-3 h_{0}^2 c_2^{(0)}
-7 h_{0} c_1^{(1)}
+2 h_{0}^2 c_2^{(1)}
&=+15 x_{0}-15 x_1\\
c_2^{(0)}&=0
\end{aligned}\right\}&&\quad\textrm{Odd}\\
\left.\begin{aligned}
c_1^{(0)}&=\alpha\hphantom{\,15 x_{0}-15 x_1}\\
c_2^{(0)}&=\beta/2
\end{aligned}\right\}&&\quad\textrm{Mix}
\end{alignat}
\end{subequations}
We then consider boundary conditions at $l=l_N$,
\begin{alignat}{3}
\textrm{Even:}\quad&&0&=\dd{s^{(N-1)}}{l}=\dd{^3s^{(N-1)}}{l^3}
&&\quad\textrm{at}\quad l=l_N\\
\textrm{Odd:}\quad&&0&=s^{(N-1)}=\dd{^2s^{(N-1)}}{l^2}=\dd{^4s^{(N-1)}}{l^4}
&&\quad\textrm{at}\quad l=l_N\\
\textrm{Mix:}\quad&&\alpha&=\dd{s^{(N-1)}}{l},\quad \beta=\dd{^2s^{(N-1)}}{l^2}
&&\quad\textrm{at}\quad l=l_N
\end{alignat}
which also lead to two equations,
\begin{subequations}\label{eq:spline_endbc}
\begin{alignat}{2}
\left.\begin{aligned}
c_1^{(N)}&=0\\
-4 h_{N-1} {c}_1^{(N-1)}
-h_{N-1}^2 {c}_2^{(N-1)}
-6 h_{N-1} {c}_1^{(N)}
+3 h_{N-1}^2 {c}_2^{(N)}
&=10 {x}_{N-1}-10 {x}_N
\end{aligned}\right\}&&\quad\textrm{Even}\\
\left.\begin{aligned}
-7 h_{N-1} {c}_1^{(N-1)}
-2 h_{N-1}^2 {c}_2^{(N-1)}
-8 h_{N-1} {c}_1^{(N)}
+3 h_{N-1}^2 {c}_2^{(N)}
&=15 {x}_{N-1}-15 {x}_N\\
c_2^{(N)}&=0
\end{aligned}\right\}&&\quad\textrm{Odd}\\
\left.\begin{aligned}
c_1^{(N)}&=\alpha\hphantom{\!\!\!\!15 {x}_{N-1}-15 {x}_N}\\
c_2^{(N)}&=\beta/2
\end{aligned}\right\}&&\quad\textrm{Mix}
\end{alignat}
\end{subequations}
If we arrange the unknowns into a vector,
\begin{equation}
\big[c_1^{(0)},c_2^{(0)},\dots,c_1^{(j)},c_2^{(j)},\dots,c_1^{(N)},c_2^{(N)}\big]
\end{equation}
equations (\ref{eq:spline_mid}) with one of (\ref{eq:spline_beginbc})
and one of (\ref{eq:spline_endbc}) result in a $2(N+1)$-by-$2(N+1)$ system of linear equations,
which corresponds to a banded diagonal sparse matrix with at most 6 non-zero elements in each row.
The rest coefficients $\vec{c}_3^{(j)}$, $\vec{c}_4^{(j)}$ and $\vec{c}_5^{(j)}$
can be reconstruct using equation (\ref{eq:spline_c3c4c5}).

\par It is convenient to re-parametrize each local spline with a new variable $t\in[0,1]$,
\begin{equation}
\vec{s}^{(j)}(t)=\vec{x}_j+\sum_{k=1}^{5}\vec{c}^{(j)}_k  t ^k,
\quad\textrm{for}\quad j=0,\dots,N-1
\end{equation}
by redefining $\vec{c}^{(j)}_k\to \vec{c}^{(j)}_kh_j^k$.
We can construct Lagrange interpolations along the arc-length of the global spline
by the local arc-length $L_j$ and its local fraction $\xi$,
\begin{equation}
L_j = \int_{0}^1\dot{\vec{s}}^{(j)}\,\mathrm{d}t',\quad
\xi(t) =\frac{1}{L_j} \int_{0}^t\dot{\vec{s}}^{(j)}\,\mathrm{d}t',\quad
\end{equation}
Convention for normal vector $\vec{n}$,
\begin{equation}
\vec{n}=\frac{(-\dot{z},\dot{r})}{\|\dot{\vec{s}}\|}
\end{equation}

\section{Numerical approximation}
\subsection{Gaussian  quadrature}
Standard $m$-point Gauss-Legendre quadrature,
\begin{equation}
\int_0^1 f(t)\,\mathrm{d}t\approx \sum_{k=1}^{m}w_k f(x_k),\quad
\int_a^b f(t)\,\mathrm{d}t\Rightarrow
%\int_a^b =\sum_{k=1}^{m}(b-a)w_k f(a+(b-a)x_k),
\begin{aligned}
x_k&\to a+(b-a)x_k\\ w_k&\to (b-a)w_k
\end{aligned}
\end{equation}
Logarithmic-weighted Gauss quadrature,
\begin{equation}
\int_0^1 f(t) \ln t \,\mathrm{d}t\approx -\sum_{k=1}^{m}w_k^\mathrm{log} f(x_k),\quad
\int_0^1 f(t) \ln (1-t) \,\mathrm{d}t\approx -\sum_{k=1}^{m}w_k^\mathrm{log} f(1-x_k)
\end{equation}
When logarithmic singularity $\tau\in (0,1)$,
\begin{align*}
&\int_0^1 f(t) \ln |t-\tau|\,\mathrm{d}t 
=\int_0^\tau f(t) \ln (\tau - t)\,\mathrm{d}t+\int_\tau^1 f(t) \ln (t-\tau)\,\mathrm{d}t\\
=&\,\tau\int_0^1 f(\tau s) \ln (\tau - \tau s)\,\mathrm{d}s
+(1-\tau) \int_0^1 f(\tau + (1-\tau) s) \ln ( (1-\tau) s)\,\mathrm{d}s\\
=&\,\tau\int_0^1 f(\tau s) \ln (1 - s)\,\mathrm{d} s
+(1-\tau) \int_0^1 f(\tau + (1-\tau) s) \ln s\,\mathrm{d}s\\
&\quad+\tau\ln\tau\int_0^1 f(\tau s) \,\mathrm{d} s
+(1-\tau)\ln (1-\tau) \int_0^1 f(\tau + (1-\tau) s) \,\mathrm{d}s
\end{align*}
Logarithmic-weighted Gauss quadrature for singularity $\tau\in (0,1)$,
\begin{equation}\int_0^1 f(t) \ln |t-\tau|\,\mathrm{d}t 
\approx\left\{ 
\begin{aligned}
\tau\ln\tau &\sum_{k=1}^m w_k f\left[\tau x_k\right] \\
+ (1- \tau)\ln(1-\tau) &\sum_{k=1}^m w_k f\left[\tau + (1-\tau)x_k\right]\\
- \tau &\sum_{k=1}^m w_k^\mathrm{log} f\left[\tau(1 -x_k^\mathrm{log})\right]\\
- (1- \tau) &\sum_{k=1}^m w_k^\mathrm{log} f\left[\tau + (1-\tau)x_k^\mathrm{log}\right]
\end{aligned}\right.
\end{equation}

\subsection{Complete elliptic integral}
The complete elliptic integral of the first and second kind,
\begin{equation}
K(m)=\int_0^{\pi/2}\frac{\mathrm{d}\theta}{\sqrt{1-m \sin^2 \theta}},\quad
E(m)=\int_0^{\pi/2}\sqrt{1-m \sin^2 \theta}\,\mathrm{d}\theta
\end{equation}
We can approximate $K(m)$ and $E(m)$ with 
\begin{equation}
K(m)\approx P_K(m) -\ln(1-m) Q_K(m),\quad
E(m)\approx P_E(m) -\ln(1-m) Q_E(m) 
\end{equation}
where $P_K$, $P_E$, $Q_K$, $Q_E$ are tenth order polynomials.
Associate Legendre polynomial of $1/2$ order can be conveniently expressed as,
\begin{equation}
P_{1/2}(x)=\frac{2}{\pi}\left\{2 E\left(\frac{1-x}{2}\right)-K\left(\frac{1-x}{2}\right)\right\}
\end{equation}
From the recursion relations,
\begin{equation}
\begin{aligned}
(1+x^2)\dd{P_\ell(x)}{x}&=(\ell+1) xP_\ell(x)-(\ell+1)P_{\ell+1}(x)\\
\dd{K}{m}&=\frac{E(m)}{2m(1-m)}-\frac{K(m)}{2m}\\
\dd{E}{m}&=\frac{1}{2m}\left[E(m)- K(m)\right]
\end{aligned}
\end{equation}
we can bootstrap associate Legendre polynomials of higher orders, $3/2$ , $5/2$ , ...,
The form is again 
\begin{equation}
P_\ell (x)= c_\ell \left\{P_{\ell, E}(x)E\left(\frac{1-x}{2}\right) +   P_{\ell, K}(x)K\left(\frac{1-x}{2}\right)\right\}
\end{equation}
where $c_\ell$ is some constant. $P_{\ell, E}(x)$ and $P_{\ell, E}(x)$ are polynomials of $(\ell - 1/2)$ order
whose coefficients are computed symbolically and given in Table \ref{tab:LegendrePolyCoef}.
\begin{table}
  \begin{center}
\def~{\hphantom{0}}
  \begin{tabular}{c@{\hskip 20pt}l@{\hskip 20pt} r|l}
$\ell$&$c_\ell$  & $P_{\ell, E}(x)$   &   $P_{\ell, K}(x)$  \\[3pt]
${1}/{2}$&${2}/{\pi }$ & $2$ & $-1$\\[2pt]
${3}/{2}$&${1}/{3 \pi }$ & $16 x$ & $-2 (4 x+1)$\\ [2pt]
${5}/{2}$&${1}/{15 \pi }$ & $4 \left(32 x^2-9\right)$ & $-2 \left(32 x^2+8 x-9\right)$\\[2pt]
${7}/{2}$&${1}/{105 \pi }$ & $64 x \left(24 x^2-13\right)$ & $2 \left(-384 x^3-96 x^2+208 x+25\right)$\\[2pt]
${9}/{2}$&${1}/{315 \pi }$ & $8192 x^4-6528 x^2+588$ & $-2 \left(2048 x^4+512 x^3-1632 x^2-264 x+147\right)$
  \end{tabular}
  \caption{Coefficients of associate Legendre polynomials of $\ell = 1/2$ family}
  \label{tab:LegendrePolyCoef}
  \end{center}
\end{table}
\section{Axisymmetric boundary integral}
Axisymmetric Green's function $G(\vec{\eta}';\vec{\eta})$,
\begin{align}
G(\vec{\eta}';\vec{\eta})&=\frac{1}{\pi\sqrt{a+b}}K(m)\\
\pd{G(\vec{\eta}';\vec{\eta})}{\vec{n}}&=
\frac{1}{\pi r\sqrt{a+b}}\bigg\{
\left[\frac{n_r}{2}+\frac{\vec{n}\cdot(\vec{x}'-\vec{x})}{\|\vec{x}'-\vec{x}\|^2}r\right]E(m)
-\frac{n_r}{2}K(m)\bigg\}
\end{align}
Auxiliary variables,
\begin{equation}
a=r'^2+r^2+(z'-z)^2,\quad b = 2 r' r,\quad m =\frac{2 b}{a+b}
\end{equation}
Note $\eta\to\eta'$ implies $m\to 1$.
Axisymmetric boundary integral equation,
\begin{equation}
\frac{\beta(\vec{\eta}')}{2\pi}\phi(\vec{\eta}')=\int_{\gamma}
\Big\{
G(\vec{\eta}';\vec{\eta})\pd{\phi}{\vec{n}}(\vec{\eta})-
\phi(\vec{\eta})\pd{G(\vec{\eta}';\vec{\eta})}{\vec{n}}(\vec{\eta})
\Big\}
 r \,\mathrm{d}\gamma(\vec{\eta})
\end{equation}
Lagrange interpolation along arc-length,
\begin{equation}
\phi(\vec{\eta})\approx\sum_e \sum_k p_{e_k}\mathcal{N}^{(e)}_k(\xi),\quad
\pd{\phi}{\vec{n}}(\vec{\eta})\approx\sum_e \sum_kq_{e_k}\mathcal{N}^{(e)}_k(\xi)
\end{equation}
where $\mathcal{N}_k$ is the Lagrange basis of the local fractional arc-length variable $\xi$,
\begin{equation}
\mathcal{N}_0=(1 - \xi) (1 - 2\xi),\quad
\mathcal{N}_1=4 \xi (1 - \xi) ,\quad
\mathcal{N}_2=\xi (2 \xi- 1)
\end{equation}
\subsection{Single layer integral}
Single-layer integral reduces to a summation of local integrals over local arc $\gamma_e$ of element $e$,
\begin{align}
\nonumber\int_{\gamma}G(\vec{\eta}';\vec{\eta})
\pd{\phi}{\vec{n}}(\vec{\eta}) r \,\mathrm{d}\gamma(\vec{\eta})
&\approx\sum_e \sum_k q_{e_k}
\int_{\gamma_e}G(\vec{\eta}';\vec{\eta})\mathcal{N}^{(e)}_k(\xi) r \,\mathrm{d}\gamma(\vec{\eta})\\
&=\sum_e \sum_k q_{e_k}
\int_0^1\frac{ r J }{\pi \sqrt{a+b}}\mathcal{N}^{(e)}_k(\xi)K(m)\,\mathrm{d}t
\end{align}
When singularity $\vec{\eta}'\in \gamma_e$ is located at $t=\tau \in[0,1]$,
\begin{align*}
\int_0^1\frac{ r J }{\pi \sqrt{a+b}}\mathcal{N}^{(e)}_kK(m)\,\mathrm{d}t
=&\int_0^1\frac{ r J }{\pi \sqrt{a+b}}\mathcal{N}^{(e)}_k
\left[P_K -Q_K\ln\frac{1-m}{(t-\tau)^2}-2 Q_K\ln |t-\tau| \right]\,\mathrm{d}t
\end{align*}
Introduce helper functions,
\begin{align}
f_K^\mathrm{single}(t)&= \frac{ r J }{\pi \sqrt{a+b}} \\
R_{K/E}(m,\tau)&= P_{K/E}(m) -Q_{K/E}(m) \ln\frac{1-m}{(t-\tau)^2}
\end{align}
If we define integrand $I^{(o)}$ for source location $\vec{\eta}'$ outside of $\gamma_e$
and $I^{(i)}$ for $\vec{\eta}'$ interior to $\gamma_e$,
\begin{align}
I^{(o)}(t) &=  f_K^\mathrm{single}(t) K(m)\mathcal{N}^{(e)}_k(\xi)\\
I^{(i)}(t,\tau) &= f_K^\mathrm{single}(t)R_K(m,\tau)\mathcal{N}^{(e)}_k(\xi)\\
I^{(i)}_\mathrm{log}(t) &=-2 f_K^\mathrm{single} (t)Q_K(m) \mathcal{N}^{(e)}_k(\xi)
\end{align}
then the local single-layer integral can be compactly represented
with a regular part and a logarithmic-singular one,
\begin{align}
\vec{\eta}' \not\in \gamma_e:\,\,\textrm{compute}&\int_0^1I^{(o)}(t)\,\mathrm{d}t\\
\vec{\eta}' \textrm{ at } t= 0:\,\,\textrm{compute}&\int_0^1I^{(i)}(t,0)\,\mathrm{d}t+
\int_0^1I^{(i)}_\mathrm{log}(t)\ln t\,\mathrm{d}t\\
\vec{\eta}' \textrm{ at } t= \tau:\,\,\textrm{compute}&\int_0^1I^{(i)}(t,\tau)\,\mathrm{d}t+
\int_0^1I^{(i)}_\mathrm{log}(t)\ln|\tau-t|\,\mathrm{d}t\\
\vec{\eta}' \textrm{ at } t= 1:\,\,\textrm{compute}&\int_0^1I^{(i)}(t,1)\,\mathrm{d}t+
\int_0^1I^{(i)}_\mathrm{log}(t)\ln(1-t)\,\mathrm{d}t
\end{align}


\subsection{Double layer integral}
Similar to singular-layer integral, double-layer integral reduces to a summation of local integrals over local arc $\gamma_e$ of element $e$,
\begin{align}
&\nonumber\int_{\gamma}\pd{G(\vec{\eta}';\vec{\eta})}{\vec{n}}\phi(\vec{\eta}) r \,\mathrm{d}\gamma(\vec{\eta})
\approx\sum_e \sum_k p_{e_k}
\int_{\gamma_e}\pd{G(\vec{\eta}';\vec{\eta})}{\vec{n}}\mathcal{N}^{(e)}_k(\xi) r \,\mathrm{d}\gamma(\vec{\eta})\\
=&\nonumber\,\sum_e \sum_k p_{e_k}
\int_0^1\mathcal{N}^{(e)}_k(\xi)
\frac{J}{\pi \sqrt{a+b}}\bigg\{
\left[\frac{n_r}{2}+\frac{\vec{n}\cdot(\vec{x}'-\vec{x})}{\|\vec{x}'-\vec{x}\|^2}r\right]E(m)
-\frac{n_r}{2}K(m)\bigg\}
\,\mathrm{d}t\\
=&\,\sum_e \sum_k p_{e_k}
\int_0^1
\frac{\mathcal{N}^{(e)}_k(\xi)}{\pi \sqrt{a+b}}\bigg\{
\left[\frac{\dot{r}(z'-z)-\dot{z}(r'-r)}{(r'-r)^2+(z'-z)^2}r
-\frac{\dot{z}}{2}\right]E(m)
+\frac{\dot{z}}{2}K(m)\bigg\}
\,\mathrm{d}t
\end{align}
Technically we don't need to treat integrand involving elliptic integral of the second kind $E(m)$.
However, convergence rate of standard Gauss-Legendre quadrature  depends magnitude of derivatives.
\begin{equation}
f_E^\mathrm{double}=\frac{1}{\pi \sqrt{a+b}}
\left[\frac{\dot{r}(z'-z)-\dot{z}(r'-r)}{a-b}r
-\frac{\dot{z}}{2}\right],\quad
f_K^\mathrm{double}=\frac{1}{\pi \sqrt{a+b}}
\frac{\dot{z}}{2}
\end{equation}
If we define integrand $I^{(o)}$ for source location $\vec{\eta}'$ outside of $\gamma_e$
and $I^{(i)}$ for $\vec{\eta}'$ interior to $\gamma_e$,
\begin{align}
I^{(o)}(t) &= \left[ f_E^\mathrm{double} E(m)+f_K^\mathrm{double} K(m)\right]\mathcal{N}^{(e)}_k(\xi)\\
I^{(i)}(t,\tau) &= \left[f_E^\mathrm{double}R_E(m,\tau)+ f_K^\mathrm{double}R_K(m,\tau)\right]\mathcal{N}^{(e)}_k(\xi)\\
I^{(i)}_\mathrm{log}(t) &=-2 f_E^\mathrm{double}Q_E(m) -2 f_K^\mathrm{double}Q_K(m) 
\end{align}
When $\vec{\eta'}$ is located the symmetry axis where $r'=0$,
we can simplify the elliptic integrals,
\begin{align}
\int_{\gamma}G(\vec{\eta}';\vec{\eta})
\pd{\phi}{\vec{n}}(\vec{\eta}) r \,\mathrm{d}\gamma(\vec{\eta})
&\approx\sum_e \sum_k q_{e_k}
\int_0^1\mathcal{N}^{(e)}_k(\xi)r\frac{   J }{
2 \sqrt{r+(z-z')^2}}\,\mathrm{d}t\\
\int_{\gamma}\pd{G(\vec{\eta}';\vec{\eta})}{\vec{n}}\phi(\vec{\eta}) r \,\mathrm{d}\gamma(\vec{\eta})
&\approx\sum_e \sum_k p_{e_k}
\int_0^1\mathcal{N}^{(e)}_k(\xi)r \frac{  
\dot{z}r+\dot{r}(z-z')
}{
2\big[r^2+(z'-z)^2\big]^{3/2}
}\,\mathrm{d}t
\end{align}
The above integrands have well-defined limit as $\vec{\eta}\to\vec{\eta}'$
assuming $\ddot{z}/\dot{r}$ is finite at $r=0$.

\subsection{Assembly}
\begin{table}
  \begin{center}
\def~{\hphantom{0}}
  \begin{tabular}{rl|r}
Element order & $o$\\
Number of elements & $n$\\
Number of nodes & $ n \times o +1$\\
Nodal index of source point & $ i$\\
Elemental index of receiving point & $ j$\\
Local node index & $k$ & $0\le k \le o$\\
Indices of elements $\ni$ $i$-th node if $i \textrm{ mod }  o =0$
& $ \lfloor i / o\rfloor-1$, $ \lfloor i / o\rfloor$& check $ 0\le$ id $\le n-1$\\
Indices of elements $\ni$  $i$-th node  if $i \textrm{ mod }  o \neq 0$
& $ \lfloor i / o\rfloor-1$& check $ 0\le$ id $\le n-1$\\
Nodal indices of $j$-th element  & $j \times o +  k$ & $0\le k \le o$\\
  \end{tabular}
  \caption{Index involved}
  \label{tab:kd}
  \end{center}
\end{table}

\begin{algorithm}
  \caption{Counting mismatches between two packed strings
    \label{alg:packed-dna-hamming}}
  \begin{algorithmic}[1]
%\algloop{For}
    \Function{Distance}{$x$, $e$}
    \For{$ 0\le i \le N_x - 1$ } \hfill\text{\# We can parallelize this loop}
    \If{$i \textrm{ mod } o = 0$}     
\State    $I_\mathrm{lower}$  =    $I_\mathrm{upper}$  
          \Else
\State              $I_\mathrm{lower} = \lfloor i/o\rfloor - 1$,\quad
              $I_\mathrm{upper} = \lfloor i/o\rfloor$  
          \EndIf
    \For{$ 0\le i \le N_x - 1$ } 
    
    \EndFor
    
    
    \EndFor
    \EndFunction
  \end{algorithmic}
\end{algorithm}

\begin{equation}
(\mat{B}+\mat{D})\vec{p} =\mat{H}\vec{p} = \mat{S}\vec{q}
\end{equation}
Consider two adjacent boundaries $\gamma_0$ and $\gamma_1$.
We have a block matrix equation,
\begin{equation}
\begin{bmatrix}
\mat{H}_{00}&\mat{H}_{01}\\\mat{H}_{10}&\mat{H}_{11}
\end{bmatrix}
\begin{bmatrix}
\vec{p}_0\\\vec{p}_1
\end{bmatrix}
=\begin{bmatrix}
\mat{S}_{00}&\mat{S}_{01}\\\mat{S}_{10}&\mat{S}_{11}
\end{bmatrix}
\begin{bmatrix}
\vec{q}_0\\\vec{q}_1
\end{bmatrix}
\end{equation}
We have two identical rows in $\mat{H}$ and $\mat{S}$ 
and extra equation for continuity of potential,
\begin{equation}\left.\begin{aligned}
\mathrm{row}_{-1}(\mat{H}_{00})\cdot\vec{p}_0
+\mathrm{row}_{-1}(\mat{H}_{01})\cdot\vec{p}_1
&=\mathrm{row}_{-1}(\mat{S}_{00})\cdot\vec{q}_0
+\mathrm{row}_{-1}(\mat{S}_{01})\cdot\vec{q}_1\\
\mathrm{row}_{0}(\mat{H}_{10})\cdot\vec{p}_0
+\mathrm{row}_{0}(\mat{H}_{11})\cdot\vec{p}_1
&=\mathrm{row}_{0}(\mat{S}_{10})\cdot\vec{q}_0
+\mathrm{row}_{0}(\mat{S}_{11})\cdot\vec{q}_1\\
(p_0)_{-1}-(p_1)_{0}&=0
\end{aligned}\right\}\end{equation}
We simply replace one of identical rows with the continuity condition,
\begin{equation}\begin{aligned}
\mathrm{row}_{-1}(\mat{H}_{00}) &= [0, \,\cdots ,0 , 1],&
\mathrm{row}_{-1}(\mat{H}_{01}) &= [-1,0, \,\cdots ,0 ]\\
\mathrm{row}_{-1}(\mat{S}_{00}) &= [0, \,\cdots ,0 , 0],&
\mathrm{row}_{-1}(\mat{S}_{01}) &= [0,0, \,\cdots ,0 ]
\end{aligned}\end{equation}
For mixed-type boundary condition, i.e., Dirichlet and Neumann on different segments of the boundary,
we simply rearrange the block matrices,
\begin{equation}
\begin{bmatrix}
\mat{H}_{00}&-\mat{S}_{01}\\\mat{H}_{10}&-\mat{S}_{11}
\end{bmatrix}
\begin{bmatrix}
\vec{p}_0\\\vec{q}_1
\end{bmatrix}
=\begin{bmatrix}
\mat{S}_{00}&-\mat{H}_{01}\\\mat{S}_{10}&-\mat{H}_{11}
\end{bmatrix}
\begin{bmatrix}
\vec{q}_0\\\vec{p}_1
\end{bmatrix}
\end{equation}
Note the continuity between discrete potential vectors $\vec{p}_0$ and $\vec{p}_1$
is still implied.

\subsection{Benchmark}
We validate our implementation of boundary integral solver for test problems posed 
on a smooth boundary $\gamma$ and on a piecewise smooth boundary $\gamma_0\cup\gamma_1$.
The latter contains a geometric discontinuity of a 90$^\circ$ corner
at the transition point between $\gamma_0$ and $\gamma_1$.
Consider the following parametrisation of the boundaries,
\begin{align}
\gamma_0&=\left\{
2 \Big(1+\frac{1}{4}\cos(8\theta - \pi)\Big)
\begin{pmatrix}
\cos\theta\\
\sin\theta
\end{pmatrix}
\Bigm|  \theta \in [0,\pi/2]\right\}\\
\gamma_1&=
\left\{
\begin{pmatrix}
t\\0
\end{pmatrix}
\Bigm|  \theta \in [0,3/2]\right\}
\end{align}
The general form of axisymmetric harmonic potential can be constructed from Legendre polynomial $P_\ell$ of order $\ell$.
We consider the interior problem for the potential $\phi$,
\begin{equation}\label{eq:test_parametrisation}
\phi = (r^2+z^2)P_2\Big(\frac{z}{\sqrt{r^2+z^2}}\Big),\quad
\pd{\phi}{\vec{n}} = \vec{n}\cdot (-r , 2 z )
\end{equation}
If we prescribe the potential value of $\phi$ on $\gamma_0$ and 
its normal flux $\partial \phi /\partial \vec{n}$ on $\gamma_1$,
our solver is expected to reproduce $\partial \phi /\partial \vec{n}$ on $\gamma_0$
and $\phi$ on $\gamma_1$,
\begin{equation}\label{eq:test_problem}
\textrm{Test problem:}\quad\textrm{given}\left\{
\begin{aligned}
\phi &\textrm{ on }\gamma_0\\
\partial \phi /\partial \vec{n}&\textrm{ on }\gamma_1
\end{aligned}\right.,
\,\,\textrm{find}\left\{
\begin{aligned}
\partial \phi /\partial \vec{n}&\textrm{ on }\gamma_0\\
\phi &\textrm{ on }\gamma_1
\end{aligned}\right.
\end{equation}
In addition for a plane curve given parametrically as $\gamma(t) = (r(t), z(t))$,
the total curvature $2\kappa$ of the surface of revolution obtained by rotating curve $\gamma$
about the $z$-axis is given by,
\begin{equation}
2\kappa= \frac{\dot{r}\ddot{z} - \dot{z}\ddot{r}}{(\dot{r}^2+\dot{z}^2)^{\frac{3}{2}}}
+\frac{1}{r}\frac{\dot{z}}{\sqrt{\dot{r}^2+\dot{z}^2}}
\end{equation}
We verify the accuracy of quintic spline interpolation against the analytic form derived from
the parametrisation (\ref{eq:test_parametrisation}). All errors between numerical and analytical solutions
are measured in the $l^\infty$-norm,
\begin{align}
\|\textrm{err}_\mathrm{d}\|_\infty &= 
\max_{\vec{x}_i\in \gamma_1}{\lvert p^\mathrm{num}_i-\phi(\vec{x}_i)\rvert}\\
\|\textrm{err}_\mathrm{n}\|_\infty &= 
\max_{\vec{x}_i\in \gamma_0}{\lvert q^\mathrm{num}_i-\partial\phi/\partial\vec{n}(\vec{x}_i)\rvert}\\
\|\textrm{err}_\mathrm{c}\|_\infty &= 
\max_{\vec{x}_i\in \gamma_0}{\lvert \kappa^\mathrm{spline}_i-\kappa(\vec{x}_i)\rvert}
\end{align}
In figure \ref{fig:test_result}(a) and \ref{fig:test_result}(b) 
we plot these errors against the total number of degree of freedom (DOF) used by the solver.
\begin{figure}
  \centering
  \includegraphics[height =  5.2cm]{test.pdf}% Images in 100% size
  \caption{
Error convergence  between numerical and analytic solutions measured in $l^\infty$-norm against degrees of freedom (DOF) used for the test problem (\ref{eq:test_problem}):
(a) Boundaries $\gamma_0$ (solid) and $\gamma_1$ (dashed) given by parametrisation (\ref{eq:test_parametrisation}).
(b) Convergence of total curvature ($\blacktriangledown$) on $\gamma_0$;
convergence of Neumann data on $\gamma_0$ with linear ($\CIRCLE $) and quadratic ($\blacksquare$) shape functions;
convergence of Dirichlet data on $\gamma_1$ with linear ($\ocircle $) and quadratic ($\Box$) shape functions;
linear fit of the last five points of each error convergence (inset).
  }
\label{fig:test_result}
\end{figure}

\section{Self-similar Taylor cone}
In the last section we outline the numerical procedure to solve Laplace equation subject to mixed boundary conditions.
Initial guess is a $C^3$ continuous function $f_\mathrm{guess}$,
\begin{equation}
f_\mathrm{guess}(r)=\left\{\begin{aligned}
&f_0r+f_2r^2+f_4r^4+f_6r^6,&&\textrm{if }r\le r_c\\
&c_0 r + \frac{c_1}{\sqrt{r}} +\frac{c_3}{r^{7/2}}+\frac{c_4}{r^{5}},&&\textrm{if }r>r_c
\end{aligned}\right.
\end{equation}
where $r_c$ is the connection point where continuities up to the third derivative are enforced.
Given a reasonable $c_1$, function $f_\mathrm{guess}(r)$ agrees with analytic prediction in the far-field.
The effect of varying $r_c$ and $c_1$ is illustrated in figure \ref{fig:initialGuess}.
\begin{figure}
  \centering
  \includegraphics[height =  4.2cm]{fig02.pdf}% Images in 100% size
  \caption{$C^3$ function $f_\mathrm{guess}(r)$: increasing $r_c$ (right) and $c_1$ (left)  }
\label{fig:initialGuess}
\end{figure}




























\bibliographystyle{jfm}
% Note the spaces between the initials
\bibliography{taylorCone}

\end{document}


%%%%%%%%%%%%%%%%%%%%%%%%%%%%%%
\begin{document}
\maketitle
\tableofcontents
\begin{abstract}
Numeric scheme \citep{Taylor64}
\end{abstract}

\begin{keywords}
electrohydrodynamics, free surface flow, surface singularity \comment{(Authors should
not enter keywords on the manuscript)}
\end{keywords}

\section{Quintic spline}
Given $N$ marker points $\{\vec{x}_0,\dots,\vec{x}_N\}$,
we first compute $(N-1)$ chord lengths $\{h_0,\dots,h_{N-1}\}$
by euclidean distance between adjacent marker points,
\begin{equation}
h_j=\|\vec{x}_{j+1}-\vec{x}_{j}\|_2\quad\textrm{for}\quad j=0,\dots,N-1
\end{equation}
Then we  introduce $N$ chord coordinates $\{l_0,\dots,l_N\}$ with $l_0=0$ and
\begin{equation}
l_j=\sum_{k=0}^{j-1}h_k\quad\textrm{for}\quad j=1,\dots,N
\end{equation}
Global spline is a collection of local splines $\{\vec{s}^{(0)},\dots,\vec{s}^{(N-1)}\}$
where each $\vec{s}^{(j)}$ is a fifth degree polynomial defined on the interval $l\in[l_j,l_{j+1}]$,
\begin{equation}
\vec{s}^{(j)}(l)=\vec{x}_j+\sum_{k=1}^{5}\vec{c}^{(j)}_k(l-l_j)^k,
\quad\textrm{for}\quad j=0,\dots,N-1
\end{equation}
Continuity between neighboring splines is satisfied up to the fourth derivative,
\begin{equation}
\left.\dd{^{p}\vec{s}^{(j)}}{l^p}\right|_{l_{j+1}}=
\left.\dd{^{p}\vec{s}^{(j+1)}}{l^p}\right|_{l_{j+1}},
\quad\textrm{for}\quad p=0,\dots,4
\end{equation}
The continuity conditions yield five equations
\begin{equation}\left.
\begin{aligned}
\vec{c}^{(j-1)}_3h_{j-1}^3+\vec{c}^{(j-1)}_4h_{j-1}^4+\vec{c}^{(j-1)}_5h_{j-1}^5
&=\vec{x}_j-\vec{x}_{j-1}-\vec{c}^{(j-1)}_1h_{j-1}-\vec{c}^{(j-1)}_2h_{j-1}^2\\
3\vec{c}^{(j-1)}_3h_{j-1}^2+4\vec{c}^{(j-1)}_4h_{j-1}^3+5\vec{c}^{(j-1)}_5h_{j-1}^4
&=-\vec{c}^{(j-1)}_1-2\vec{c}^{(j-1)}_2h_{j-1}+\vec{c}_1^{(j)}\\
6\vec{c}^{(j-1)}_3h_{j-1}+12\vec{c}^{(j-1)}_4h_{j-1}^2+20\vec{c}^{(j-1)}_5h_{j-1}^3
&=-2\vec{c}^{(j-1)}_2+2\vec{c}_2^{(j)}\\
\vec{c}_3^{(j-1)}+4 h_{j-1} \vec{c}_4^{(j-1)}+10 h_{j-1}^2 \vec{c}_5^{(j-1)}
&=\vec{c}_3^{(j)}\\
\vec{c}_4^{(j-1)}+5 h_{j-1} \vec{c}_5^{(j-1)}
&=\vec{c}_4^{(j)}
\end{aligned}\right\}
\end{equation}
We first solve for $\vec{c}^{(j-1)}_3$, $\vec{c}^{(j-1)}_4$ and $\vec{c}^{(j-1)}_5$,
\begin{equation}\label{eq:spline_c3c4c5}\left.
\begin{aligned}
h^3_{j-1}\vec{c}^{(j-1)}_3&=\\
&-6 h_{j-1} \vec{c}_1^{(j-1)}
-3 h_{j-1}^2 \vec{c}_2^{(j-1)}
-4 h_{j-1} \vec{c}_1^{(j)}
+h_{j-1}^2 \vec{c}_2^{(j)}
+10( \vec{x}_j- \vec{x}_{j-1})\\
h^4_{j-1}\vec{c}^{(j-1)}_4&=\\
&+8 h_{j-1} \vec{c}_1^{(j-1)}
+3 h_{j-1}^2 \vec{c}_2^{(j-1)}
+7 h_{j-1} \vec{c}_1^{(j)}
-2 h_{j-1}^2 \vec{c}_2^{(j)}
-15 (\vec{x}_j- \vec{x}_{j-1})\\
h^5_{j-1}\vec{c}^{(j-1)}_5&=\\
&-3 h_{j-1} \vec{c}_1^{(j-1)}
-\phantom{1}h_{j-1}^2 \vec{c}_2^{(j-1)}
-3 h_{j-1} \vec{c}_1^{(j)}
+\phantom{1}h_{j-1}^2 \vec{c}_2^{(j)}
+\phantom{1}6( \vec{x}_j- \vec{x}_{j-1})
\end{aligned}\right\}
\end{equation}
Define $\lambda=h_j/h_{j-1}$. We have $2(N-1)$ equations for $j=1,\dots, N-1$,
\begin{equation}\label{eq:spline_mid}
\left.\begin{aligned}
10\left[+ \lambda^3 \vec{x}_{j-1}- (1+\lambda^3) \vec{x}_j+\vec{x}_{j+1}\right]=
&&4 h_{j} \lambda^2 \vec{c}_1^{(j-1)}
&&+6 h_{j} (\lambda^2-1) \vec{c}_1^{(j)}
&&-4 h_{j} \vec{c}_1^{(j+1)}\\
&&+h_{j}^2 \lambda \vec{c}_2^{(j-1)}
&&-3 h_{j}^2 (1+\lambda) \vec{c}_2^{(j)}
&&+h_{j}^2 \vec{c}_2^{(j+1)}
\\
15 \left[- \lambda^4 \vec{x}_{j-1}+ (\lambda^4-1) \vec{x}_j+\vec{x}_{j+1}\right]=
&&7 h_{j} \lambda^3 \vec{c}_1^{(j-1)}
&&+8 h_{j} (1+\lambda^3) \vec{c}_1^{(j)}
&&+7 h_{j} \vec{c}_1^{(j+1)}\\
&&+2 h_{j}^2 \lambda^2 \vec{c}_2^{(j-1)}
&&+3h_{j}^2 (1- \lambda^2) \vec{c}_2^{(j)}
&&-2 h_{j}^2 \vec{c}_2^{(j+1)}
\end{aligned}\right\}
\end{equation}
for $2(N+1)$ unknowns,
 $\{\vec{c}_1^{(0)},\dots,\vec{c}_1^{(N)}\}$ and  $\{\vec{c}_2^{(0)},\dots,\vec{c}_2^{(N)}\}$.
The imposed boundary conditions at the beginning and end of the global spline
produce four additional equations for
$\{\vec{c}_1^{(0)}, \vec{c}_2^{(0)},\vec{c}_1^{(N)}, \vec{c}_2^{(N)}\}$.
Note we have implicitly introduced a ghost spline $\vec{s}^{(N)}$
which satisfies all continuity conditions with $\vec{s}^{(N-1)}$.
The purpose of $\vec{s}^{(N)}$ is to deal with boundary conditions at the end of spline.

\par In general there are three types of boundary conditions at each end.
Let $x$, $s$ and  ${c}^{(j)}$ be one of the scalar components of vector $\vec{x}_j$,
spline $\vec{s}^{(j)}$ and coefficient $\vec{c}^{(j)}$.
We first consider boundary conditions at $l=l_0$,
\begin{alignat}{3}
\textrm{Even:}\quad&&0&=\dd{s^{(0)}}{l}=\dd{^3s^{(0)}}{l^3}&&\quad\textrm{at}\quad l=l_0\\
\textrm{Odd:}\quad&&0&=s^{(0)}=\dd{^2s^{(0)}}{l^2}=\dd{^4s^{(0)}}{l^4}&&\quad\textrm{at}\quad l=l_0\\
\textrm{Mix:}\quad&&\alpha&=\dd{s^{(0)}}{l},\quad \beta=\dd{^2s^{(0)}}{l^2}&&\quad\textrm{at}\quad l=l_0
\end{alignat}
which lead to two equations,
\begin{subequations}\label{eq:spline_beginbc}
\begin{alignat}{2}
\left.\begin{aligned}
c_1^{(0)}&=0\\
6 h_{0} c_1^{(0)}
+3 h_{0}^2 c_2^{(0)}
+4 h_{0} c_1^{(1)}
-h_{0}^2 c_2^{(1)}
&=-10 x_{0}+10 x_1
\end{aligned}\right\}&&\quad\textrm{Even}\\
\left.\begin{aligned}
-8 h_{0} c_1^{(0)}
-3 h_{0}^2 c_2^{(0)}
-7 h_{0} c_1^{(1)}
+2 h_{0}^2 c_2^{(1)}
&=+15 x_{0}-15 x_1\\
c_2^{(0)}&=0
\end{aligned}\right\}&&\quad\textrm{Odd}\\
\left.\begin{aligned}
c_1^{(0)}&=\alpha\hphantom{\,15 x_{0}-15 x_1}\\
c_2^{(0)}&=\beta/2
\end{aligned}\right\}&&\quad\textrm{Mix}
\end{alignat}
\end{subequations}
We then consider boundary conditions at $l=l_N$,
\begin{alignat}{3}
\textrm{Even:}\quad&&0&=\dd{s^{(N-1)}}{l}=\dd{^3s^{(N-1)}}{l^3}
&&\quad\textrm{at}\quad l=l_N\\
\textrm{Odd:}\quad&&0&=s^{(N-1)}=\dd{^2s^{(N-1)}}{l^2}=\dd{^4s^{(N-1)}}{l^4}
&&\quad\textrm{at}\quad l=l_N\\
\textrm{Mix:}\quad&&\alpha&=\dd{s^{(N-1)}}{l},\quad \beta=\dd{^2s^{(N-1)}}{l^2}
&&\quad\textrm{at}\quad l=l_N
\end{alignat}
which also lead to two equations,
\begin{subequations}\label{eq:spline_endbc}
\begin{alignat}{2}
\left.\begin{aligned}
c_1^{(N)}&=0\\
-4 h_{N-1} {c}_1^{(N-1)}
-h_{N-1}^2 {c}_2^{(N-1)}
-6 h_{N-1} {c}_1^{(N)}
+3 h_{N-1}^2 {c}_2^{(N)}
&=10 {x}_{N-1}-10 {x}_N
\end{aligned}\right\}&&\quad\textrm{Even}\\
\left.\begin{aligned}
-7 h_{N-1} {c}_1^{(N-1)}
-2 h_{N-1}^2 {c}_2^{(N-1)}
-8 h_{N-1} {c}_1^{(N)}
+3 h_{N-1}^2 {c}_2^{(N)}
&=15 {x}_{N-1}-15 {x}_N\\
c_2^{(N)}&=0
\end{aligned}\right\}&&\quad\textrm{Odd}\\
\left.\begin{aligned}
c_1^{(N)}&=\alpha\hphantom{\!\!\!\!15 {x}_{N-1}-15 {x}_N}\\
c_2^{(N)}&=\beta/2
\end{aligned}\right\}&&\quad\textrm{Mix}
\end{alignat}
\end{subequations}
If we arrange the unknowns into a vector,
\begin{equation}
\big[c_1^{(0)},c_2^{(0)},\dots,c_1^{(j)},c_2^{(j)},\dots,c_1^{(N)},c_2^{(N)}\big]
\end{equation}
equations (\ref{eq:spline_mid}) with one of (\ref{eq:spline_beginbc})
and one of (\ref{eq:spline_endbc}) result in a $2(N+1)$-by-$2(N+1)$ system of linear equations,
which corresponds to a banded diagonal sparse matrix with at most 6 non-zero elements in each row.
The rest coefficients $\vec{c}_3^{(j)}$, $\vec{c}_4^{(j)}$ and $\vec{c}_5^{(j)}$
can be reconstruct using equation (\ref{eq:spline_c3c4c5}).

\par It is convenient to re-parametrize each local spline with a new variable $t\in[0,1]$,
\begin{equation}
\vec{s}^{(j)}(t)=\vec{x}_j+\sum_{k=1}^{5}\vec{c}^{(j)}_k  t ^k,
\quad\textrm{for}\quad j=0,\dots,N-1
\end{equation}
by redefining $\vec{c}^{(j)}_k\to \vec{c}^{(j)}_kh_j^k$.
We can construct Lagrange interpolations along the arc-length of the global spline
by the local arc-length $L_j$ and its local fraction $\xi$,
\begin{equation}
L_j = \int_{0}^1\dot{\vec{s}}^{(j)}\,\mathrm{d}t',\quad
\xi(t) =\frac{1}{L_j} \int_{0}^t\dot{\vec{s}}^{(j)}\,\mathrm{d}t',\quad
\end{equation}
Convention for normal vector $\vec{n}$,
\begin{equation}
\vec{n}=\frac{(-\dot{z},\dot{r})}{\|\dot{\vec{s}}\|}
\end{equation}

\section{Numerical approximation}
\subsection{Gaussian  quadrature}
Standard $m$-point Gauss-Legendre quadrature,
\begin{equation}
\int_0^1 f(t)\,\mathrm{d}t\approx \sum_{k=1}^{m}w_k f(x_k),\quad
\int_a^b f(t)\,\mathrm{d}t\Rightarrow
%\int_a^b =\sum_{k=1}^{m}(b-a)w_k f(a+(b-a)x_k),
\begin{aligned}
x_k&\to a+(b-a)x_k\\ w_k&\to (b-a)w_k
\end{aligned}
\end{equation}
Logarithmic-weighted Gauss quadrature,
\begin{equation}
\int_0^1 f(t) \ln t \,\mathrm{d}t\approx -\sum_{k=1}^{m}w_k^\mathrm{log} f(x_k),\quad
\int_0^1 f(t) \ln (1-t) \,\mathrm{d}t\approx -\sum_{k=1}^{m}w_k^\mathrm{log} f(1-x_k)
\end{equation}
When logarithmic singularity $\tau\in (0,1)$,
\begin{align*}
&\int_0^1 f(t) \ln |t-\tau|\,\mathrm{d}t 
=\int_0^\tau f(t) \ln (\tau - t)\,\mathrm{d}t+\int_\tau^1 f(t) \ln (t-\tau)\,\mathrm{d}t\\
=&\,\tau\int_0^1 f(\tau s) \ln (\tau - \tau s)\,\mathrm{d}s
+(1-\tau) \int_0^1 f(\tau + (1-\tau) s) \ln ( (1-\tau) s)\,\mathrm{d}s\\
=&\,\tau\int_0^1 f(\tau s) \ln (1 - s)\,\mathrm{d} s
+(1-\tau) \int_0^1 f(\tau + (1-\tau) s) \ln s\,\mathrm{d}s\\
&\quad+\tau\ln\tau\int_0^1 f(\tau s) \,\mathrm{d} s
+(1-\tau)\ln (1-\tau) \int_0^1 f(\tau + (1-\tau) s) \,\mathrm{d}s
\end{align*}
Logarithmic-weighted Gauss quadrature for singularity $\tau\in (0,1)$,
\begin{equation}\int_0^1 f(t) \ln |t-\tau|\,\mathrm{d}t 
\approx\left\{ 
\begin{aligned}
\tau\ln\tau &\sum_{k=1}^m w_k f\left[\tau x_k\right] \\
+ (1- \tau)\ln(1-\tau) &\sum_{k=1}^m w_k f\left[\tau + (1-\tau)x_k\right]\\
- \tau &\sum_{k=1}^m w_k^\mathrm{log} f\left[\tau(1 -x_k^\mathrm{log})\right]\\
- (1- \tau) &\sum_{k=1}^m w_k^\mathrm{log} f\left[\tau + (1-\tau)x_k^\mathrm{log}\right]
\end{aligned}\right.
\end{equation}

\subsection{Complete elliptic integral}
The complete elliptic integral of the first and second kind,
\begin{equation}
K(m)=\int_0^{\pi/2}\frac{\mathrm{d}\theta}{\sqrt{1-m \sin^2 \theta}},\quad
E(m)=\int_0^{\pi/2}\sqrt{1-m \sin^2 \theta}\,\mathrm{d}\theta
\end{equation}
We can approximate $K(m)$ and $E(m)$ with 
\begin{equation}
K(m)\approx P_K(m) -\ln(1-m) Q_K(m),\quad
E(m)\approx P_E(m) -\ln(1-m) Q_E(m) 
\end{equation}
where $P_K$, $P_E$, $Q_K$, $Q_E$ are tenth order polynomials.
Associate Legendre polynomial of $1/2$ order can be conveniently expressed as,
\begin{equation}
P_{1/2}(x)=\frac{2}{\pi}\left\{2 E\left(\frac{1-x}{2}\right)-K\left(\frac{1-x}{2}\right)\right\}
\end{equation}
From the recursion relations,
\begin{equation}
\begin{aligned}
(1+x^2)\dd{P_\ell(x)}{x}&=(\ell+1) xP_\ell(x)-(\ell+1)P_{\ell+1}(x)\\
\dd{K}{m}&=\frac{E(m)}{2m(1-m)}-\frac{K(m)}{2m}\\
\dd{E}{m}&=\frac{1}{2m}\left[E(m)- K(m)\right]
\end{aligned}
\end{equation}
we can bootstrap associate Legendre polynomials of higher orders, $3/2$ , $5/2$ , ...,
The form is again 
\begin{equation}
P_\ell (x)= c_\ell \left\{P_{\ell, E}(x)E\left(\frac{1-x}{2}\right) +   P_{\ell, K}(x)K\left(\frac{1-x}{2}\right)\right\}
\end{equation}
where $c_\ell$ is some constant. $P_{\ell, E}(x)$ and $P_{\ell, E}(x)$ are polynomials of $(\ell - 1/2)$ order
whose coefficients are computed symbolically and given in Table \ref{tab:LegendrePolyCoef}.
\begin{table}
  \begin{center}
\def~{\hphantom{0}}
  \begin{tabular}{c@{\hskip 20pt}l@{\hskip 20pt} r|l}
$\ell$&$c_\ell$  & $P_{\ell, E}(x)$   &   $P_{\ell, K}(x)$  \\[3pt]
${1}/{2}$&${2}/{\pi }$ & $2$ & $-1$\\[2pt]
${3}/{2}$&${1}/{3 \pi }$ & $16 x$ & $-2 (4 x+1)$\\ [2pt]
${5}/{2}$&${1}/{15 \pi }$ & $4 \left(32 x^2-9\right)$ & $-2 \left(32 x^2+8 x-9\right)$\\[2pt]
${7}/{2}$&${1}/{105 \pi }$ & $64 x \left(24 x^2-13\right)$ & $2 \left(-384 x^3-96 x^2+208 x+25\right)$\\[2pt]
${9}/{2}$&${1}/{315 \pi }$ & $8192 x^4-6528 x^2+588$ & $-2 \left(2048 x^4+512 x^3-1632 x^2-264 x+147\right)$
  \end{tabular}
  \caption{Coefficients of associate Legendre polynomials of $\ell = 1/2$ family}
  \label{tab:LegendrePolyCoef}
  \end{center}
\end{table}
\section{Axisymmetric boundary integral}
Axisymmetric Green's function $G(\vec{\eta}';\vec{\eta})$,
\begin{align}
G(\vec{\eta}';\vec{\eta})&=\frac{1}{\pi\sqrt{a+b}}K(m)\\
\pd{G(\vec{\eta}';\vec{\eta})}{\vec{n}}&=
\frac{1}{\pi r\sqrt{a+b}}\bigg\{
\left[\frac{n_r}{2}+\frac{\vec{n}\cdot(\vec{x}'-\vec{x})}{\|\vec{x}'-\vec{x}\|^2}r\right]E(m)
-\frac{n_r}{2}K(m)\bigg\}
\end{align}
Auxiliary variables,
\begin{equation}
a=r'^2+r^2+(z'-z)^2,\quad b = 2 r' r,\quad m =\frac{2 b}{a+b}
\end{equation}
Note $\eta\to\eta'$ implies $m\to 1$.
Axisymmetric boundary integral equation,
\begin{equation}
\frac{\beta(\vec{\eta}')}{2\pi}\phi(\vec{\eta}')=\int_{\gamma}
\Big\{
G(\vec{\eta}';\vec{\eta})\pd{\phi}{\vec{n}}(\vec{\eta})-
\phi(\vec{\eta})\pd{G(\vec{\eta}';\vec{\eta})}{\vec{n}}(\vec{\eta})
\Big\}
 r \,\mathrm{d}\gamma(\vec{\eta})
\end{equation}
Lagrange interpolation along arc-length,
\begin{equation}
\phi(\vec{\eta})\approx\sum_e \sum_k p_{e_k}\mathcal{N}^{(e)}_k(\xi),\quad
\pd{\phi}{\vec{n}}(\vec{\eta})\approx\sum_e \sum_kq_{e_k}\mathcal{N}^{(e)}_k(\xi)
\end{equation}
where $\mathcal{N}_k$ is the Lagrange basis of the local fractional arc-length variable $\xi$,
\begin{equation}
\mathcal{N}_0=(1 - \xi) (1 - 2\xi),\quad
\mathcal{N}_1=4 \xi (1 - \xi) ,\quad
\mathcal{N}_2=\xi (2 \xi- 1)
\end{equation}
\subsection{Single layer integral}
Single-layer integral reduces to a summation of local integrals over local arc $\gamma_e$ of element $e$,
\begin{align}
\nonumber\int_{\gamma}G(\vec{\eta}';\vec{\eta})
\pd{\phi}{\vec{n}}(\vec{\eta}) r \,\mathrm{d}\gamma(\vec{\eta})
&\approx\sum_e \sum_k q_{e_k}
\int_{\gamma_e}G(\vec{\eta}';\vec{\eta})\mathcal{N}^{(e)}_k(\xi) r \,\mathrm{d}\gamma(\vec{\eta})\\
&=\sum_e \sum_k q_{e_k}
\int_0^1\frac{ r J }{\pi \sqrt{a+b}}\mathcal{N}^{(e)}_k(\xi)K(m)\,\mathrm{d}t
\end{align}
When singularity $\vec{\eta}'\in \gamma_e$ is located at $t=\tau \in[0,1]$,
\begin{align*}
\int_0^1\frac{ r J }{\pi \sqrt{a+b}}\mathcal{N}^{(e)}_kK(m)\,\mathrm{d}t
=&\int_0^1\frac{ r J }{\pi \sqrt{a+b}}\mathcal{N}^{(e)}_k
\left[P_K -Q_K\ln\frac{1-m}{(t-\tau)^2}-2 Q_K\ln |t-\tau| \right]\,\mathrm{d}t
\end{align*}
Introduce helper functions,
\begin{align}
f_K^\mathrm{single}(t)&= \frac{ r J }{\pi \sqrt{a+b}} \\
R_{K/E}(m,\tau)&= P_{K/E}(m) -Q_{K/E}(m) \ln\frac{1-m}{(t-\tau)^2}
\end{align}
If we define integrand $I^{(o)}$ for source location $\vec{\eta}'$ outside of $\gamma_e$
and $I^{(i)}$ for $\vec{\eta}'$ interior to $\gamma_e$,
\begin{align}
I^{(o)}(t) &=  f_K^\mathrm{single}(t) K(m)\mathcal{N}^{(e)}_k(\xi)\\
I^{(i)}(t,\tau) &= f_K^\mathrm{single}(t)R_K(m,\tau)\mathcal{N}^{(e)}_k(\xi)\\
I^{(i)}_\mathrm{log}(t) &=-2 f_K^\mathrm{single} (t)Q_K(m) \mathcal{N}^{(e)}_k(\xi)
\end{align}
then the local single-layer integral can be compactly represented
with a regular part and a logarithmic-singular one,
\begin{align}
\vec{\eta}' \not\in \gamma_e:\,\,\textrm{compute}&\int_0^1I^{(o)}(t)\,\mathrm{d}t\\
\vec{\eta}' \textrm{ at } t= 0:\,\,\textrm{compute}&\int_0^1I^{(i)}(t,0)\,\mathrm{d}t+
\int_0^1I^{(i)}_\mathrm{log}(t)\ln t\,\mathrm{d}t\\
\vec{\eta}' \textrm{ at } t= \tau:\,\,\textrm{compute}&\int_0^1I^{(i)}(t,\tau)\,\mathrm{d}t+
\int_0^1I^{(i)}_\mathrm{log}(t)\ln|\tau-t|\,\mathrm{d}t\\
\vec{\eta}' \textrm{ at } t= 1:\,\,\textrm{compute}&\int_0^1I^{(i)}(t,1)\,\mathrm{d}t+
\int_0^1I^{(i)}_\mathrm{log}(t)\ln(1-t)\,\mathrm{d}t
\end{align}


\subsection{Double layer integral}
Similar to singular-layer integral, double-layer integral reduces to a summation of local integrals over local arc $\gamma_e$ of element $e$,
\begin{align}
&\nonumber\int_{\gamma}\pd{G(\vec{\eta}';\vec{\eta})}{\vec{n}}\phi(\vec{\eta}) r \,\mathrm{d}\gamma(\vec{\eta})
\approx\sum_e \sum_k p_{e_k}
\int_{\gamma_e}\pd{G(\vec{\eta}';\vec{\eta})}{\vec{n}}\mathcal{N}^{(e)}_k(\xi) r \,\mathrm{d}\gamma(\vec{\eta})\\
=&\nonumber\,\sum_e \sum_k p_{e_k}
\int_0^1\mathcal{N}^{(e)}_k(\xi)
\frac{J}{\pi \sqrt{a+b}}\bigg\{
\left[\frac{n_r}{2}+\frac{\vec{n}\cdot(\vec{x}'-\vec{x})}{\|\vec{x}'-\vec{x}\|^2}r\right]E(m)
-\frac{n_r}{2}K(m)\bigg\}
\,\mathrm{d}t\\
=&\,\sum_e \sum_k p_{e_k}
\int_0^1
\frac{\mathcal{N}^{(e)}_k(\xi)}{\pi \sqrt{a+b}}\bigg\{
\left[\frac{\dot{r}(z'-z)-\dot{z}(r'-r)}{(r'-r)^2+(z'-z)^2}r
-\frac{\dot{z}}{2}\right]E(m)
+\frac{\dot{z}}{2}K(m)\bigg\}
\,\mathrm{d}t
\end{align}
Technically we don't need to treat integrand involving elliptic integral of the second kind $E(m)$.
However, convergence rate of standard Gauss-Legendre quadrature  depends magnitude of derivatives.
\begin{equation}
f_E^\mathrm{double}=\frac{1}{\pi \sqrt{a+b}}
\left[\frac{\dot{r}(z'-z)-\dot{z}(r'-r)}{a-b}r
-\frac{\dot{z}}{2}\right],\quad
f_K^\mathrm{double}=\frac{1}{\pi \sqrt{a+b}}
\frac{\dot{z}}{2}
\end{equation}
If we define integrand $I^{(o)}$ for source location $\vec{\eta}'$ outside of $\gamma_e$
and $I^{(i)}$ for $\vec{\eta}'$ interior to $\gamma_e$,
\begin{align}
I^{(o)}(t) &= \left[ f_E^\mathrm{double} E(m)+f_K^\mathrm{double} K(m)\right]\mathcal{N}^{(e)}_k(\xi)\\
I^{(i)}(t,\tau) &= \left[f_E^\mathrm{double}R_E(m,\tau)+ f_K^\mathrm{double}R_K(m,\tau)\right]\mathcal{N}^{(e)}_k(\xi)\\
I^{(i)}_\mathrm{log}(t) &=-2 f_E^\mathrm{double}Q_E(m) -2 f_K^\mathrm{double}Q_K(m) 
\end{align}
When $\vec{\eta'}$ is located the symmetry axis where $r'=0$,
we can simplify the elliptic integrals,
\begin{align}
\int_{\gamma}G(\vec{\eta}';\vec{\eta})
\pd{\phi}{\vec{n}}(\vec{\eta}) r \,\mathrm{d}\gamma(\vec{\eta})
&\approx\sum_e \sum_k q_{e_k}
\int_0^1\mathcal{N}^{(e)}_k(\xi)r\frac{   J }{
2 \sqrt{r+(z-z')^2}}\,\mathrm{d}t\\
\int_{\gamma}\pd{G(\vec{\eta}';\vec{\eta})}{\vec{n}}\phi(\vec{\eta}) r \,\mathrm{d}\gamma(\vec{\eta})
&\approx\sum_e \sum_k p_{e_k}
\int_0^1\mathcal{N}^{(e)}_k(\xi)r \frac{  
\dot{z}r+\dot{r}(z-z')
}{
2\big[r^2+(z'-z)^2\big]^{3/2}
}\,\mathrm{d}t
\end{align}
The above integrands have well-defined limit as $\vec{\eta}\to\vec{\eta}'$
assuming $\ddot{z}/\dot{r}$ is finite at $r=0$.

\subsection{Assembly}
\begin{table}
  \begin{center}
\def~{\hphantom{0}}
  \begin{tabular}{rl|r}
Element order & $o$\\
Number of elements & $n$\\
Number of nodes & $ n \times o +1$\\
Nodal index of source point & $ i$\\
Elemental index of receiving point & $ j$\\
Local node index & $k$ & $0\le k \le o$\\
Indices of elements $\ni$ $i$-th node if $i \textrm{ mod }  o =0$
& $ \lfloor i / o\rfloor-1$, $ \lfloor i / o\rfloor$& check $ 0\le$ id $\le n-1$\\
Indices of elements $\ni$  $i$-th node  if $i \textrm{ mod }  o \neq 0$
& $ \lfloor i / o\rfloor-1$& check $ 0\le$ id $\le n-1$\\
Nodal indices of $j$-th element  & $j \times o +  k$ & $0\le k \le o$\\
  \end{tabular}
  \caption{Index involved}
  \label{tab:kd}
  \end{center}
\end{table}

\begin{algorithm}
  \caption{Counting mismatches between two packed strings
    \label{alg:packed-dna-hamming}}
  \begin{algorithmic}[1]
%\algloop{For}
    \Function{Distance}{$x$, $e$}
    \For{$ 0\le i \le N_x - 1$ } \hfill\text{\# We can parallelize this loop}
    \If{$i \textrm{ mod } o = 0$}     
\State    $I_\mathrm{lower}$  =    $I_\mathrm{upper}$  
          \Else
\State              $I_\mathrm{lower} = \lfloor i/o\rfloor - 1$,\quad
              $I_\mathrm{upper} = \lfloor i/o\rfloor$  
          \EndIf
    \For{$ 0\le i \le N_x - 1$ } 
    
    \EndFor
    
    
    \EndFor
    \EndFunction
  \end{algorithmic}
\end{algorithm}

\begin{equation}
(\mat{B}+\mat{D})\vec{p} =\mat{H}\vec{p} = \mat{S}\vec{q}
\end{equation}
Consider two adjacent boundaries $\gamma_0$ and $\gamma_1$.
We have a block matrix equation,
\begin{equation}
\begin{bmatrix}
\mat{H}_{00}&\mat{H}_{01}\\\mat{H}_{10}&\mat{H}_{11}
\end{bmatrix}
\begin{bmatrix}
\vec{p}_0\\\vec{p}_1
\end{bmatrix}
=\begin{bmatrix}
\mat{S}_{00}&\mat{S}_{01}\\\mat{S}_{10}&\mat{S}_{11}
\end{bmatrix}
\begin{bmatrix}
\vec{q}_0\\\vec{q}_1
\end{bmatrix}
\end{equation}
We have two identical rows in $\mat{H}$ and $\mat{S}$ 
and extra equation for continuity of potential,
\begin{equation}\left.\begin{aligned}
\mathrm{row}_{-1}(\mat{H}_{00})\cdot\vec{p}_0
+\mathrm{row}_{-1}(\mat{H}_{01})\cdot\vec{p}_1
&=\mathrm{row}_{-1}(\mat{S}_{00})\cdot\vec{q}_0
+\mathrm{row}_{-1}(\mat{S}_{01})\cdot\vec{q}_1\\
\mathrm{row}_{0}(\mat{H}_{10})\cdot\vec{p}_0
+\mathrm{row}_{0}(\mat{H}_{11})\cdot\vec{p}_1
&=\mathrm{row}_{0}(\mat{S}_{10})\cdot\vec{q}_0
+\mathrm{row}_{0}(\mat{S}_{11})\cdot\vec{q}_1\\
(p_0)_{-1}-(p_1)_{0}&=0
\end{aligned}\right\}\end{equation}
We simply replace one of identical rows with the continuity condition,
\begin{equation}\begin{aligned}
\mathrm{row}_{-1}(\mat{H}_{00}) &= [0, \,\cdots ,0 , 1],&
\mathrm{row}_{-1}(\mat{H}_{01}) &= [-1,0, \,\cdots ,0 ]\\
\mathrm{row}_{-1}(\mat{S}_{00}) &= [0, \,\cdots ,0 , 0],&
\mathrm{row}_{-1}(\mat{S}_{01}) &= [0,0, \,\cdots ,0 ]
\end{aligned}\end{equation}
For mixed-type boundary condition, i.e., Dirichlet and Neumann on different segments of the boundary,
we simply rearrange the block matrices,
\begin{equation}
\begin{bmatrix}
\mat{H}_{00}&-\mat{S}_{01}\\\mat{H}_{10}&-\mat{S}_{11}
\end{bmatrix}
\begin{bmatrix}
\vec{p}_0\\\vec{q}_1
\end{bmatrix}
=\begin{bmatrix}
\mat{S}_{00}&-\mat{H}_{01}\\\mat{S}_{10}&-\mat{H}_{11}
\end{bmatrix}
\begin{bmatrix}
\vec{q}_0\\\vec{p}_1
\end{bmatrix}
\end{equation}
Note the continuity between discrete potential vectors $\vec{p}_0$ and $\vec{p}_1$
is still implied.

\subsection{Benchmark}
We validate our implementation of boundary integral solver for test problems posed 
on a smooth boundary $\gamma$ and on a piecewise smooth boundary $\gamma_0\cup\gamma_1$.
The latter contains a geometric discontinuity of a 90$^\circ$ corner
at the transition point between $\gamma_0$ and $\gamma_1$.
Consider the following parametrisation of the boundaries,
\begin{align}
\gamma_0&=\left\{
2 \Big(1+\frac{1}{4}\cos(8\theta - \pi)\Big)
\begin{pmatrix}
\cos\theta\\
\sin\theta
\end{pmatrix}
\Bigm|  \theta \in [0,\pi/2]\right\}\\
\gamma_1&=
\left\{
\begin{pmatrix}
t\\0
\end{pmatrix}
\Bigm|  \theta \in [0,3/2]\right\}
\end{align}
The general form of axisymmetric harmonic potential can be constructed from Legendre polynomial $P_\ell$ of order $\ell$.
We consider the interior problem for the potential $\phi$,
\begin{equation}\label{eq:test_parametrisation}
\phi = (r^2+z^2)P_2\Big(\frac{z}{\sqrt{r^2+z^2}}\Big),\quad
\pd{\phi}{\vec{n}} = \vec{n}\cdot (-r , 2 z )
\end{equation}
If we prescribe the potential value of $\phi$ on $\gamma_0$ and 
its normal flux $\partial \phi /\partial \vec{n}$ on $\gamma_1$,
our solver is expected to reproduce $\partial \phi /\partial \vec{n}$ on $\gamma_0$
and $\phi$ on $\gamma_1$,
\begin{equation}\label{eq:test_problem}
\textrm{Test problem:}\quad\textrm{given}\left\{
\begin{aligned}
\phi &\textrm{ on }\gamma_0\\
\partial \phi /\partial \vec{n}&\textrm{ on }\gamma_1
\end{aligned}\right.,
\,\,\textrm{find}\left\{
\begin{aligned}
\partial \phi /\partial \vec{n}&\textrm{ on }\gamma_0\\
\phi &\textrm{ on }\gamma_1
\end{aligned}\right.
\end{equation}
In addition for a plane curve given parametrically as $\gamma(t) = (r(t), z(t))$,
the total curvature $2\kappa$ of the surface of revolution obtained by rotating curve $\gamma$
about the $z$-axis is given by,
\begin{equation}
2\kappa= \frac{\dot{r}\ddot{z} - \dot{z}\ddot{r}}{(\dot{r}^2+\dot{z}^2)^{\frac{3}{2}}}
+\frac{1}{r}\frac{\dot{z}}{\sqrt{\dot{r}^2+\dot{z}^2}}
\end{equation}
We verify the accuracy of quintic spline interpolation against the analytic form derived from
the parametrisation (\ref{eq:test_parametrisation}). All errors between numerical and analytical solutions
are measured in the $l^\infty$-norm,
\begin{align}
\|\textrm{err}_\mathrm{d}\|_\infty &= 
\max_{\vec{x}_i\in \gamma_1}{\lvert p^\mathrm{num}_i-\phi(\vec{x}_i)\rvert}\\
\|\textrm{err}_\mathrm{n}\|_\infty &= 
\max_{\vec{x}_i\in \gamma_0}{\lvert q^\mathrm{num}_i-\partial\phi/\partial\vec{n}(\vec{x}_i)\rvert}\\
\|\textrm{err}_\mathrm{c}\|_\infty &= 
\max_{\vec{x}_i\in \gamma_0}{\lvert \kappa^\mathrm{spline}_i-\kappa(\vec{x}_i)\rvert}
\end{align}
In figure \ref{fig:test_result}(a) and \ref{fig:test_result}(b) 
we plot these errors against the total number of degree of freedom (DOF) used by the solver.
\begin{figure}
  \centering
  \includegraphics[height =  5.2cm]{test.pdf}% Images in 100% size
  \caption{
Error convergence  between numerical and analytic solutions measured in $l^\infty$-norm against degrees of freedom (DOF) used for the test problem (\ref{eq:test_problem}):
(a) Boundaries $\gamma_0$ (solid) and $\gamma_1$ (dashed) given by parametrisation (\ref{eq:test_parametrisation}).
(b) Convergence of total curvature ($\blacktriangledown$) on $\gamma_0$;
convergence of Neumann data on $\gamma_0$ with linear ($\CIRCLE $) and quadratic ($\blacksquare$) shape functions;
convergence of Dirichlet data on $\gamma_1$ with linear ($\ocircle $) and quadratic ($\Box$) shape functions;
linear fit of the last five points of each error convergence (inset).
  }
\label{fig:test_result}
\end{figure}

\section{Self-similar Taylor cone}
In the last section we outline the numerical procedure to solve Laplace equation subject to mixed boundary conditions.
Initial guess is a $C^3$ continuous function $f_\mathrm{guess}$,
\begin{equation}
f_\mathrm{guess}(r)=\left\{\begin{aligned}
&f_0r+f_2r^2+f_4r^4+f_6r^6,&&\textrm{if }r\le r_c\\
&c_0 r + \frac{c_1}{\sqrt{r}} +\frac{c_3}{r^{7/2}}+\frac{c_4}{r^{5}},&&\textrm{if }r>r_c
\end{aligned}\right.
\end{equation}
where $r_c$ is the connection point where continuities up to the third derivative are enforced.
Given a reasonable $c_1$, function $f_\mathrm{guess}(r)$ agrees with analytic prediction in the far-field.
The effect of varying $r_c$ and $c_1$ is illustrated in figure \ref{fig:initialGuess}.
\begin{figure}
  \centering
  \includegraphics[height =  4.2cm]{fig02.pdf}% Images in 100% size
  \caption{$C^3$ function $f_\mathrm{guess}(r)$: increasing $r_c$ (right) and $c_1$ (left)  }
\label{fig:initialGuess}
\end{figure}




























\bibliographystyle{jfm}
% Note the spaces between the initials
\bibliography{taylorCone}

\end{document}


%%%%%%%%%%%%%%%%%%%%%%%%%%%%%%
\begin{document}
\maketitle
\tableofcontents
\begin{abstract}
<<<<<<< HEAD
Numeric scheme %\citep{Taylor64}
=======
Numeric scheme \citep{Taylor64}
>>>>>>> 261252fba4bf040dc7cc35b34ba8820bbadbae79
\end{abstract}

\begin{keywords}
electrohydrodynamics, free surface flow, surface singularity \comment{(Authors should
not enter keywords on the manuscript)}
\end{keywords}
<<<<<<< HEAD
\section{Numerical approximation}
\subsection{Quintic spline}
The accuracy of the numerical solution to boundary integral equation is directly affected by
the precision and smoothness of the underlying geometry, in this case a planar curve.
Numerically we compute a quintic spline curve that interpolates through a discrete set of marker points
that is at least continuous up to fourth derivative \citep{Mund75}.

Consider $N + 1$ marker points $\{\vec{x}_0,\dots,\vec{x}_{N}\}$.
A natural choice for the intrinsic coordinate that can be used for interpolation
is the chord length,
\begin{equation}
h_j=\|\vec{x}_{j+1}-\vec{x}_{j}\|\quad\textrm{for}\quad j=0,\dots,N-1
\end{equation}
where $h_j$ is the the euclidean distance between adjacent marker points $\vec{x}j$ and $\vec{x}_{j+1}$.
Then we  introduce $N+1$ chord coordinates $\{l_0,\dots,l_N\}$,
\begin{equation}
l_0=0,\quad l_j=\sum_{k=0}^{j-1}h_k\quad\textrm{for}\quad j=1,\dots,N
\end{equation}
Global spline $\vec{s}(l)$ is a collection of local splines $\{\vec{s}^{(0)},\dots,\vec{s}^{(N-1)}\}$
=======

\section{Quintic spline}
Given $N$ marker points $\{\vec{x}_0,\dots,\vec{x}_N\}$,
we first compute $(N-1)$ chord lengths $\{h_0,\dots,h_{N-1}\}$
by euclidean distance between adjacent marker points,
\begin{equation}
h_j=\|\vec{x}_{j+1}-\vec{x}_{j}\|_2\quad\textrm{for}\quad j=0,\dots,N-1
\end{equation}
Then we  introduce $N$ chord coordinates $\{l_0,\dots,l_N\}$ with $l_0=0$ and
\begin{equation}
l_j=\sum_{k=0}^{j-1}h_k\quad\textrm{for}\quad j=1,\dots,N
\end{equation}
Global spline is a collection of local splines $\{\vec{s}^{(0)},\dots,\vec{s}^{(N-1)}\}$
>>>>>>> 261252fba4bf040dc7cc35b34ba8820bbadbae79
where each $\vec{s}^{(j)}$ is a fifth degree polynomial defined on the interval $l\in[l_j,l_{j+1}]$,
\begin{equation}
\vec{s}^{(j)}(l)=\vec{x}_j+\sum_{k=1}^{5}\vec{c}^{(j)}_k(l-l_j)^k,
\quad\textrm{for}\quad j=0,\dots,N-1
\end{equation}
Continuity between neighboring splines is satisfied up to the fourth derivative,
\begin{equation}
\left.\dd{^{p}\vec{s}^{(j)}}{l^p}\right|_{l_{j+1}}=
\left.\dd{^{p}\vec{s}^{(j+1)}}{l^p}\right|_{l_{j+1}},
\quad\textrm{for}\quad p=0,\dots,4
\end{equation}
<<<<<<< HEAD
These continuity conditions yield five equations
\begin{subequations}\label{eq:spline_cont}%\left.
\begin{align}\nonumber
\vec{c}^{(j-1)}_3h_{j-1}^3+\vec{c}^{(j-1)}_4h_{j-1}^4+\vec{c}^{(j-1)}_5h_{j-1}^5
&=\vec{x}_j-\vec{x}_{j-1}\\%-\vec{c}^{(j-1)}_1h_{j-1}-\vec{c}^{(j-1)}_2h_{j-1}^2\\
&\phantom{=}-\vec{c}^{(j-1)}_1h_{j-1}-\vec{c}^{(j-1)}_2h_{j-1}^2\\
=======
The continuity conditions yield five equations
\begin{equation}\left.
\begin{aligned}
\vec{c}^{(j-1)}_3h_{j-1}^3+\vec{c}^{(j-1)}_4h_{j-1}^4+\vec{c}^{(j-1)}_5h_{j-1}^5
&=\vec{x}_j-\vec{x}_{j-1}-\vec{c}^{(j-1)}_1h_{j-1}-\vec{c}^{(j-1)}_2h_{j-1}^2\\
>>>>>>> 261252fba4bf040dc7cc35b34ba8820bbadbae79
3\vec{c}^{(j-1)}_3h_{j-1}^2+4\vec{c}^{(j-1)}_4h_{j-1}^3+5\vec{c}^{(j-1)}_5h_{j-1}^4
&=-\vec{c}^{(j-1)}_1-2\vec{c}^{(j-1)}_2h_{j-1}+\vec{c}_1^{(j)}\\
6\vec{c}^{(j-1)}_3h_{j-1}+12\vec{c}^{(j-1)}_4h_{j-1}^2+20\vec{c}^{(j-1)}_5h_{j-1}^3
&=-2\vec{c}^{(j-1)}_2+2\vec{c}_2^{(j)}\\
\vec{c}_3^{(j-1)}+4 h_{j-1} \vec{c}_4^{(j-1)}+10 h_{j-1}^2 \vec{c}_5^{(j-1)}
&=\vec{c}_3^{(j)}\\
\vec{c}_4^{(j-1)}+5 h_{j-1} \vec{c}_5^{(j-1)}
&=\vec{c}_4^{(j)}
<<<<<<< HEAD
\end{align}%\right\}
\end{subequations}
We express the entire spline only in terms of the first and second coefficients 
$\{\vec{c}^{(j)}_1\}$ and $\{\vec{c}^{(j)}_2\}$.
We first solve equations 
(\ref{eq:spline_cont}$\,a$, \ref{eq:spline_cont}$\,c$, \ref{eq:spline_cont}$\,b$) for $\vec{c}^{(j-1)}_3$,
 $\vec{c}^{(j-1)}_4$ and $\vec{c}^{(j-1)}_5$,
\begin{subequations}\label{eq:spline_c3c4c5}
\begin{align}
\nonumber h^3_{j-1}\vec{c}^{(j-1)}_3=&+10( \vec{x}_j- \vec{x}_{j-1})\\
&-6 h_{j-1} \vec{c}_1^{(j-1)}
-3 h_{j-1}^2 \vec{c}_2^{(j-1)}
-4 h_{j-1} \vec{c}_1^{(j)}
+h_{j-1}^2 \vec{c}_2^{(j)}\\
\nonumber h^4_{j-1}\vec{c}^{(j-1)}_4=&-15 (\vec{x}_j- \vec{x}_{j-1})\\
&+8 h_{j-1} \vec{c}_1^{(j-1)}
+3 h_{j-1}^2 \vec{c}_2^{(j-1)}
+7 h_{j-1} \vec{c}_1^{(j)}
-2 h_{j-1}^2 \vec{c}_2^{(j)}\\
\nonumber h^5_{j-1}\vec{c}^{(j-1)}_5=&+\phantom{1}6( \vec{x}_j- \vec{x}_{j-1})\\
=======
\end{aligned}\right\}
\end{equation}
We first solve for $\vec{c}^{(j-1)}_3$, $\vec{c}^{(j-1)}_4$ and $\vec{c}^{(j-1)}_5$,
\begin{equation}\label{eq:spline_c3c4c5}\left.
\begin{aligned}
h^3_{j-1}\vec{c}^{(j-1)}_3&=\\
&-6 h_{j-1} \vec{c}_1^{(j-1)}
-3 h_{j-1}^2 \vec{c}_2^{(j-1)}
-4 h_{j-1} \vec{c}_1^{(j)}
+h_{j-1}^2 \vec{c}_2^{(j)}
+10( \vec{x}_j- \vec{x}_{j-1})\\
h^4_{j-1}\vec{c}^{(j-1)}_4&=\\
&+8 h_{j-1} \vec{c}_1^{(j-1)}
+3 h_{j-1}^2 \vec{c}_2^{(j-1)}
+7 h_{j-1} \vec{c}_1^{(j)}
-2 h_{j-1}^2 \vec{c}_2^{(j)}
-15 (\vec{x}_j- \vec{x}_{j-1})\\
h^5_{j-1}\vec{c}^{(j-1)}_5&=\\
>>>>>>> 261252fba4bf040dc7cc35b34ba8820bbadbae79
&-3 h_{j-1} \vec{c}_1^{(j-1)}
-\phantom{1}h_{j-1}^2 \vec{c}_2^{(j-1)}
-3 h_{j-1} \vec{c}_1^{(j)}
+\phantom{1}h_{j-1}^2 \vec{c}_2^{(j)}
<<<<<<< HEAD
\end{align}
\end{subequations}
Once we have $\{\vec{c}^{(j)}_1\}$ and $\{\vec{c}^{(j)}_2\}$ computed,
$\{\vec{c}^{(j)}_3\}$, $\{\vec{c}^{(j)}_4\}$ and $\{\vec{c}^{(j)}_5\}$
can be derived.



=======
+\phantom{1}6( \vec{x}_j- \vec{x}_{j-1})
\end{aligned}\right\}
\end{equation}
>>>>>>> 261252fba4bf040dc7cc35b34ba8820bbadbae79
Define $\lambda=h_j/h_{j-1}$. We have $2(N-1)$ equations for $j=1,\dots, N-1$,
\begin{equation}\label{eq:spline_mid}
\left.\begin{aligned}
10\left[+ \lambda^3 \vec{x}_{j-1}- (1+\lambda^3) \vec{x}_j+\vec{x}_{j+1}\right]=
&&4 h_{j} \lambda^2 \vec{c}_1^{(j-1)}
&&+6 h_{j} (\lambda^2-1) \vec{c}_1^{(j)}
&&-4 h_{j} \vec{c}_1^{(j+1)}\\
&&+h_{j}^2 \lambda \vec{c}_2^{(j-1)}
&&-3 h_{j}^2 (1+\lambda) \vec{c}_2^{(j)}
&&+h_{j}^2 \vec{c}_2^{(j+1)}
\\
15 \left[- \lambda^4 \vec{x}_{j-1}+ (\lambda^4-1) \vec{x}_j+\vec{x}_{j+1}\right]=
&&7 h_{j} \lambda^3 \vec{c}_1^{(j-1)}
&&+8 h_{j} (1+\lambda^3) \vec{c}_1^{(j)}
&&+7 h_{j} \vec{c}_1^{(j+1)}\\
&&+2 h_{j}^2 \lambda^2 \vec{c}_2^{(j-1)}
&&+3h_{j}^2 (1- \lambda^2) \vec{c}_2^{(j)}
&&-2 h_{j}^2 \vec{c}_2^{(j+1)}
\end{aligned}\right\}
\end{equation}
for $2(N+1)$ unknowns,
 $\{\vec{c}_1^{(0)},\dots,\vec{c}_1^{(N)}\}$ and  $\{\vec{c}_2^{(0)},\dots,\vec{c}_2^{(N)}\}$.
The imposed boundary conditions at the beginning and end of the global spline
produce four additional equations for
$\{\vec{c}_1^{(0)}, \vec{c}_2^{(0)},\vec{c}_1^{(N)}, \vec{c}_2^{(N)}\}$.
Note we have implicitly introduced a ghost spline $\vec{s}^{(N)}$
which satisfies all continuity conditions with $\vec{s}^{(N-1)}$.
The purpose of $\vec{s}^{(N)}$ is to deal with boundary conditions at the end of spline.

\par In general there are three types of boundary conditions at each end.
Let $x$, $s$ and  ${c}^{(j)}$ be one of the scalar components of vector $\vec{x}_j$,
spline $\vec{s}^{(j)}$ and coefficient $\vec{c}^{(j)}$.
We first consider boundary conditions at $l=l_0$,
\begin{alignat}{3}
\textrm{Even:}\quad&&0&=\dd{s^{(0)}}{l}=\dd{^3s^{(0)}}{l^3}&&\quad\textrm{at}\quad l=l_0\\
\textrm{Odd:}\quad&&0&=s^{(0)}=\dd{^2s^{(0)}}{l^2}=\dd{^4s^{(0)}}{l^4}&&\quad\textrm{at}\quad l=l_0\\
\textrm{Mix:}\quad&&\alpha&=\dd{s^{(0)}}{l},\quad \beta=\dd{^2s^{(0)}}{l^2}&&\quad\textrm{at}\quad l=l_0
\end{alignat}
which lead to two equations,
\begin{subequations}\label{eq:spline_beginbc}
\begin{alignat}{2}
\left.\begin{aligned}
c_1^{(0)}&=0\\
6 h_{0} c_1^{(0)}
+3 h_{0}^2 c_2^{(0)}
+4 h_{0} c_1^{(1)}
-h_{0}^2 c_2^{(1)}
&=-10 x_{0}+10 x_1
\end{aligned}\right\}&&\quad\textrm{Even}\\
\left.\begin{aligned}
-8 h_{0} c_1^{(0)}
-3 h_{0}^2 c_2^{(0)}
-7 h_{0} c_1^{(1)}
+2 h_{0}^2 c_2^{(1)}
&=+15 x_{0}-15 x_1\\
c_2^{(0)}&=0
\end{aligned}\right\}&&\quad\textrm{Odd}\\
\left.\begin{aligned}
c_1^{(0)}&=\alpha\hphantom{\,15 x_{0}-15 x_1}\\
c_2^{(0)}&=\beta/2
\end{aligned}\right\}&&\quad\textrm{Mix}
\end{alignat}
\end{subequations}
We then consider boundary conditions at $l=l_N$,
\begin{alignat}{3}
\textrm{Even:}\quad&&0&=\dd{s^{(N-1)}}{l}=\dd{^3s^{(N-1)}}{l^3}
&&\quad\textrm{at}\quad l=l_N\\
\textrm{Odd:}\quad&&0&=s^{(N-1)}=\dd{^2s^{(N-1)}}{l^2}=\dd{^4s^{(N-1)}}{l^4}
&&\quad\textrm{at}\quad l=l_N\\
\textrm{Mix:}\quad&&\alpha&=\dd{s^{(N-1)}}{l},\quad \beta=\dd{^2s^{(N-1)}}{l^2}
&&\quad\textrm{at}\quad l=l_N
\end{alignat}
which also lead to two equations,
\begin{subequations}\label{eq:spline_endbc}
\begin{alignat}{2}
\left.\begin{aligned}
c_1^{(N)}&=0\\
-4 h_{N-1} {c}_1^{(N-1)}
-h_{N-1}^2 {c}_2^{(N-1)}
-6 h_{N-1} {c}_1^{(N)}
+3 h_{N-1}^2 {c}_2^{(N)}
&=10 {x}_{N-1}-10 {x}_N
\end{aligned}\right\}&&\quad\textrm{Even}\\
\left.\begin{aligned}
-7 h_{N-1} {c}_1^{(N-1)}
-2 h_{N-1}^2 {c}_2^{(N-1)}
-8 h_{N-1} {c}_1^{(N)}
+3 h_{N-1}^2 {c}_2^{(N)}
&=15 {x}_{N-1}-15 {x}_N\\
c_2^{(N)}&=0
\end{aligned}\right\}&&\quad\textrm{Odd}\\
\left.\begin{aligned}
c_1^{(N)}&=\alpha\hphantom{\!\!\!\!15 {x}_{N-1}-15 {x}_N}\\
c_2^{(N)}&=\beta/2
\end{aligned}\right\}&&\quad\textrm{Mix}
\end{alignat}
\end{subequations}
If we arrange the unknowns into a vector,
\begin{equation}
\big[c_1^{(0)},c_2^{(0)},\dots,c_1^{(j)},c_2^{(j)},\dots,c_1^{(N)},c_2^{(N)}\big]
\end{equation}
equations (\ref{eq:spline_mid}) with one of (\ref{eq:spline_beginbc})
and one of (\ref{eq:spline_endbc}) result in a $2(N+1)$-by-$2(N+1)$ system of linear equations,
which corresponds to a banded diagonal sparse matrix with at most 6 non-zero elements in each row.
The rest coefficients $\vec{c}_3^{(j)}$, $\vec{c}_4^{(j)}$ and $\vec{c}_5^{(j)}$
can be reconstruct using equation (\ref{eq:spline_c3c4c5}).

\par It is convenient to re-parametrize each local spline with a new variable $t\in[0,1]$,
\begin{equation}
\vec{s}^{(j)}(t)=\vec{x}_j+\sum_{k=1}^{5}\vec{c}^{(j)}_k  t ^k,
\quad\textrm{for}\quad j=0,\dots,N-1
\end{equation}
by redefining $\vec{c}^{(j)}_k\to \vec{c}^{(j)}_kh_j^k$.
We can construct Lagrange interpolations along the arc-length of the global spline
by the local arc-length $L_j$ and its local fraction $\xi$,
\begin{equation}
L_j = \int_{0}^1\dot{\vec{s}}^{(j)}\,\mathrm{d}t',\quad
\xi(t) =\frac{1}{L_j} \int_{0}^t\dot{\vec{s}}^{(j)}\,\mathrm{d}t',\quad
\end{equation}
Convention for normal vector $\vec{n}$,
\begin{equation}
\vec{n}=\frac{1}{\|\dot{\vec{s}}\|}(-\dot{z},\dot{r})
\end{equation}
In addition the total curvature $2\kappa$ of the surface of revolution obtained by rotating curve $\vec{s}(t)$
about the $z$-axis is given by,
\begin{equation}
2\kappa= \frac{\dot{r}\ddot{z} - \dot{z}\ddot{r}}{(\dot{r}^2+\dot{z}^2)^{\frac{3}{2}}}
+\frac{1}{r}\frac{\dot{z}}{\sqrt{\dot{r}^2+\dot{z}^2}}
<<<<<<< HEAD
=2\frac{\ddot{z}}{\dot{r}^2} \textrm{ if } r\to 0
\end{equation}

=======
=2\frac{\ddot{z}}{\dot{r}^2} \textrm{ if on symmetry axis} 
\end{equation}
\section{Numerical approximation}
>>>>>>> 261252fba4bf040dc7cc35b34ba8820bbadbae79
\subsection{Gaussian  quadrature}
Standard $m$-point Gauss-Legendre quadrature,
\begin{equation}
\int_0^1 f(t)\,\mathrm{d}t\approx \sum_{k=1}^{m}w_k f(x_k),\quad
\int_a^b f(t)\,\mathrm{d}t\approx \sum_{k=1}^{m}(b-a)w_k f( a+(b-a)x_k)
\end{equation}
Logarithmic-weighted Gauss quadrature,
\begin{equation}
\int_0^1 f(t) \ln t \,\mathrm{d}t\approx -\sum_{k=1}^{m}w_k^\mathrm{log} f(x_k),\quad
\int_0^1 f(t) \ln (1-t) \,\mathrm{d}t\approx -\sum_{k=1}^{m}w_k^\mathrm{log} f(1-x_k)
\end{equation}
When logarithmic singularity $\tau\in (0,1)$,
\begin{align*}
&\int_0^1 f(t) \ln |t-\tau|\,\mathrm{d}t 
=\int_0^\tau f(t) \ln (\tau - t)\,\mathrm{d}t+\int_\tau^1 f(t) \ln (t-\tau)\,\mathrm{d}t\\
=&\,\tau\int_0^1 f(\tau s) \ln (\tau - \tau s)\,\mathrm{d}s
+(1-\tau) \int_0^1 f(\tau + (1-\tau) s) \ln ( (1-\tau) s)\,\mathrm{d}s\\
=&\,\tau\int_0^1 f(\tau s) \ln (1 - s)\,\mathrm{d} s
+(1-\tau) \int_0^1 f(\tau + (1-\tau) s) \ln s\,\mathrm{d}s\\
&\quad+\tau\ln\tau\int_0^1 f(\tau s) \,\mathrm{d} s
+(1-\tau)\ln (1-\tau) \int_0^1 f(\tau + (1-\tau) s) \,\mathrm{d}s
\end{align*}
Logarithmic-weighted Gauss quadrature for singularity $\tau\in (0,1)$,
\begin{equation}\int_0^1 f(t) \ln |t-\tau|\,\mathrm{d}t 
\approx\left\{ 
\begin{aligned}
\tau\ln\tau &\sum_{k=1}^m w_k f\left[\tau x_k\right] \\
+ (1- \tau)\ln(1-\tau) &\sum_{k=1}^m w_k f\left[\tau + (1-\tau)x_k\right]\\
- \tau &\sum_{k=1}^m w_k^\mathrm{log} f\left[\tau(1 -x_k^\mathrm{log})\right]\\
- (1- \tau) &\sum_{k=1}^m w_k^\mathrm{log} f\left[\tau + (1-\tau)x_k^\mathrm{log}\right]
\end{aligned}\right.
\end{equation}

\subsection{Complete elliptic integral}
The complete elliptic integral of the first and second kind,
\begin{equation}
K(m)=\int_0^{\pi/2}\frac{\mathrm{d}\theta}{\sqrt{1-m \sin^2 \theta}},\quad
E(m)=\int_0^{\pi/2}\sqrt{1-m \sin^2 \theta}\,\mathrm{d}\theta
\end{equation}
We can approximate $K(m)$ and $E(m)$ with 
\begin{equation}
<<<<<<< HEAD
K(m)\approx K_P(m) -\ln(1-m) K_Q(m),\quad
E(m)\approx E_P(m) -\ln(1-m) E_Q(m) 
\end{equation}
where $K_P$, $E_P$, $K_Q$, $E_Q$ are tenth-order polynomials.
=======
K(m)\approx P_K(m) -\ln(1-m) Q_K(m),\quad
E(m)\approx P_E(m) -\ln(1-m) Q_E(m) 
\end{equation}
where $P_K$, $P_E$, $Q_K$, $Q_E$ are tenth order polynomials.
>>>>>>> 261252fba4bf040dc7cc35b34ba8820bbadbae79
Associate Legendre polynomial of $1/2$ order can be conveniently expressed as,
\begin{equation}
P_{1/2}(x)=\frac{2}{\pi}\left\{2 E\left(\frac{1-x}{2}\right)-K\left(\frac{1-x}{2}\right)\right\}
\end{equation}
From the recursion relations,
\begin{equation}
\begin{aligned}
(1+x^2)\dd{P_\ell(x)}{x}&=(\ell+1) xP_\ell(x)-(\ell+1)P_{\ell+1}(x)\\
\dd{K}{m}&=\frac{E(m)}{2m(1-m)}-\frac{K(m)}{2m}\\
\dd{E}{m}&=\frac{1}{2m}\left[E(m)- K(m)\right]
\end{aligned}
\end{equation}
<<<<<<< HEAD
we can bootstrap higher-order associate Legendre polynomials
of $1/2+1$, $1/2+2$ , ....
=======
we can bootstrap associate Legendre polynomials of higher orders, $3/2$ , $5/2$ , ...,
The form is again 
>>>>>>> 261252fba4bf040dc7cc35b34ba8820bbadbae79
\begin{equation}
P_\ell (x)= c_\ell \left\{P_{\ell, E}(x)E\left(\frac{1-x}{2}\right) +   P_{\ell, K}(x)K\left(\frac{1-x}{2}\right)\right\}
\end{equation}
where $c_\ell$ is some constant. $P_{\ell, E}(x)$ and $P_{\ell, E}(x)$ are polynomials of $(\ell - 1/2)$ order
whose coefficients are computed symbolically and given in Table \ref{tab:LegendrePolyCoef}.
\begin{table}
  \begin{center}
\def~{\hphantom{0}}
  \begin{tabular}{c@{\hskip 20pt}l@{\hskip 20pt} r|l}
$\ell$&$c_\ell$  & $P_{\ell, E}(x)$   &   $P_{\ell, K}(x)$  \\[3pt]
${1}/{2}$&${2}/{\pi }$ & $2$ & $-1$\\[2pt]
${3}/{2}$&${1}/{3 \pi }$ & $16 x$ & $-2 (4 x+1)$\\ [2pt]
${5}/{2}$&${1}/{15 \pi }$ & $4 \left(32 x^2-9\right)$ & $-2 \left(32 x^2+8 x-9\right)$\\[2pt]
<<<<<<< HEAD
${7}/{2}$&${1}/{105 \pi }$ & $64 x \left(24 x^2-13\right)$ & $-2 \left(384 x^3+96 x^2-208 x-25\right)$\\[2pt]
=======
${7}/{2}$&${1}/{105 \pi }$ & $64 x \left(24 x^2-13\right)$ & $2 \left(-384 x^3-96 x^2+208 x+25\right)$\\[2pt]
>>>>>>> 261252fba4bf040dc7cc35b34ba8820bbadbae79
${9}/{2}$&${1}/{315 \pi }$ & $8192 x^4-6528 x^2+588$ & $-2 \left(2048 x^4+512 x^3-1632 x^2-264 x+147\right)$
  \end{tabular}
  \caption{Coefficients of associate Legendre polynomials of $\ell = 1/2$ family}
  \label{tab:LegendrePolyCoef}
  \end{center}
\end{table}
\section{Axisymmetric boundary integral}
<<<<<<< HEAD
Consider a \textit{field point} $\vec{\eta}'$ and a \textit{source point} $\vec{\eta}$ in the axisymmetric coordinate.
We introduce the axisymmetric Green's function $G(\vec{\eta}';\vec{\eta})$ by integrating out the azimuthal
dependence \citep{Lennon79},
=======
Axisymmetric Green's function $G(\vec{\eta}';\vec{\eta})$,
>>>>>>> 261252fba4bf040dc7cc35b34ba8820bbadbae79
\begin{align}
G(\vec{\eta}';\vec{\eta})&=\frac{1}{\pi\sqrt{a+b}}K(m)\\
\pd{G(\vec{\eta}';\vec{\eta})}{\vec{n}}&=
\frac{1}{\pi r\sqrt{a+b}}\bigg\{
\left[\frac{n_r}{2}+\frac{\vec{n}\cdot(\vec{x}'-\vec{x})}{\|\vec{x}'-\vec{x}\|^2}r\right]E(m)
-\frac{n_r}{2}K(m)\bigg\}
\end{align}
<<<<<<< HEAD
with auxiliary variables defined as,
\begin{equation}
a=r'^2+r^2+(z'-z)^2,\quad b = 2 r' r,\quad m =\frac{2 b}{a+b}
\end{equation}
Note $\vec{\eta}\to\vec{\eta}'$ implies $m\to 1$ which leads to a logarithmic blow-up of the Green's function
(recall elliptic integral $E(m)\sim \ln (1- m)$ as $m\to1$).


Consider a closed, simply connected region $\Omega$
and its boundary $\gamma$.
Green's third identity yields the axisymmetric boundary integral equation,
\begin{equation}\label{eq:bie}
\int_{\gamma}
=======
Auxiliary variables,
\begin{equation}
a=r'^2+r^2+(z'-z)^2,\quad b = 2 r' r,\quad m =\frac{2 b}{a+b}
\end{equation}
Note $\eta\to\eta'$ implies $m\to 1$.
Axisymmetric boundary integral equation,
\begin{equation}
\frac{\beta(\vec{\eta}')}{2\pi}\phi(\vec{\eta}')=\int_{\gamma}
>>>>>>> 261252fba4bf040dc7cc35b34ba8820bbadbae79
\Big\{
G(\vec{\eta}';\vec{\eta})\pd{\phi}{\vec{n}}(\vec{\eta})-
\phi(\vec{\eta})\pd{G(\vec{\eta}';\vec{\eta})}{\vec{n}}(\vec{\eta})
\Big\}
<<<<<<< HEAD
 r \,\mathrm{d}\gamma(\vec{\eta})=\left\{
\begin{aligned}
&\frac{\beta(\vec{\eta}')}{2\pi}\phi(\vec{\eta}')&&\textrm{ if }\vec{\eta}'\in\gamma\\
&\phi(\vec{\eta}')&&\textrm{ if }\vec{\eta}'\in\Omega
\end{aligned}\right.
\end{equation}
Given boundary conditions on $\gamma$,
we first solve the boundary integral equation 
to completely determine the value of $\partial\phi/\partial \vec{n}$
and $\phi$ on $\gamma$.
Then any field value interior to $\Omega$ can be integrated 
according to the classical form of Green's third identity.

Boundary integral equation (\ref{eq:bie}) can be numerically solved
by collocation method \citep{Kress13}.
We seek an approximate solution from a finite dimensional subspace by
requiring equation (\ref{eq:bie}) is satisfied only at a finite number of \textit{collocation points}.
=======
 r \,\mathrm{d}\gamma(\vec{\eta})
\end{equation}
>>>>>>> 261252fba4bf040dc7cc35b34ba8820bbadbae79
Lagrange interpolation along arc-length,
\begin{equation}
\phi(\vec{\eta})\approx\sum_e \sum_k p_{e_k}\mathcal{N}^{(e)}_k(\xi),\quad
\pd{\phi}{\vec{n}}(\vec{\eta})\approx\sum_e \sum_kq_{e_k}\mathcal{N}^{(e)}_k(\xi)
\end{equation}
<<<<<<< HEAD
where $\mathcal{N}_k$ is the Lagrange basis of the local fractional arc-length variable $\xi$.
We are interested in the first and second order basis which provide sufficient accuracy for our purpose.
\begin{align}
\textrm{1st order:}&
&\mathcal{N}_0&=1 - \xi,
&\mathcal{N}_1 &= \xi
& \\
\textrm{2nd order:}&
&\mathcal{N}_0&=(1 - \xi) (1 - 2\xi),
&\mathcal{N}_1&=4 \xi (1 - \xi) ,
&\mathcal{N}_2=\xi (2 \xi- 1)
\end{align}
=======
where $\mathcal{N}_k$ is the Lagrange basis of the local fractional arc-length variable $\xi$,
\begin{equation}
\mathcal{N}_0=(1 - \xi) (1 - 2\xi),\quad
\mathcal{N}_1=4 \xi (1 - \xi) ,\quad
\mathcal{N}_2=\xi (2 \xi- 1)
\end{equation}
>>>>>>> 261252fba4bf040dc7cc35b34ba8820bbadbae79
\subsection{Single layer integral}
Single-layer integral reduces to a summation of local integrals over local arc $\gamma_e$ of element $e$,
\begin{align}
\nonumber\int_{\gamma}G(\vec{\eta}';\vec{\eta})
\pd{\phi}{\vec{n}}(\vec{\eta}) r \,\mathrm{d}\gamma(\vec{\eta})
&\approx\sum_e \sum_k q_{e_k}
\int_{\gamma_e}G(\vec{\eta}';\vec{\eta})\mathcal{N}^{(e)}_k(\xi) r \,\mathrm{d}\gamma(\vec{\eta})\\
&=\sum_e \sum_k q_{e_k}
\int_0^1\frac{ r J }{\pi \sqrt{a+b}}\mathcal{N}^{(e)}_k(\xi)K(m)\,\mathrm{d}t
\end{align}
When singularity $\vec{\eta}'\in \gamma_e$ is located at $t=\tau \in[0,1]$,
\begin{align*}
\int_0^1\frac{ r J }{\pi \sqrt{a+b}}\mathcal{N}^{(e)}_kK(m)\,\mathrm{d}t
=&\int_0^1\frac{ r J }{\pi \sqrt{a+b}}\mathcal{N}^{(e)}_k
\left[P_K -Q_K\ln\frac{1-m}{(t-\tau)^2}-2 Q_K\ln |t-\tau| \right]\,\mathrm{d}t
\end{align*}
Introduce helper functions,
\begin{align}
f_K^\mathrm{single}(t)&= \frac{ r J }{\pi \sqrt{a+b}} \\
R_{K/E}(m,\tau)&= P_{K/E}(m) -Q_{K/E}(m) \ln\frac{1-m}{(t-\tau)^2}
\end{align}
If we define integrand $I^{(o)}$ for source location $\vec{\eta}'$ outside of $\gamma_e$
and $I^{(i)}$ for $\vec{\eta}'$ interior to $\gamma_e$,
\begin{align}
I^{(o)}(t) &=  f_K^\mathrm{single}(t) K(m)\mathcal{N}^{(e)}_k(\xi)\\
I^{(i)}(t,\tau) &= f_K^\mathrm{single}(t)R_K(m,\tau)\mathcal{N}^{(e)}_k(\xi)\\
I^{(i)}_\mathrm{log}(t) &=-2 f_K^\mathrm{single} (t)Q_K(m) \mathcal{N}^{(e)}_k(\xi)
\end{align}
then the local single-layer integral can be compactly represented
with a regular part and a logarithmic-singular one,
\begin{align}
\vec{\eta}' \not\in \gamma_e:\,\,\textrm{compute}&\int_0^1I^{(o)}(t)\,\mathrm{d}t\\
\vec{\eta}' \textrm{ at } t= 0:\,\,\textrm{compute}&\int_0^1I^{(i)}(t,0)\,\mathrm{d}t+
\int_0^1I^{(i)}_\mathrm{log}(t)\ln t\,\mathrm{d}t\\
\vec{\eta}' \textrm{ at } t= \tau:\,\,\textrm{compute}&\int_0^1I^{(i)}(t,\tau)\,\mathrm{d}t+
\int_0^1I^{(i)}_\mathrm{log}(t)\ln|\tau-t|\,\mathrm{d}t\\
\vec{\eta}' \textrm{ at } t= 1:\,\,\textrm{compute}&\int_0^1I^{(i)}(t,1)\,\mathrm{d}t+
\int_0^1I^{(i)}_\mathrm{log}(t)\ln(1-t)\,\mathrm{d}t
\end{align}


\subsection{Double layer integral}
Similar to singular-layer integral, double-layer integral reduces to a summation of local integrals over local arc $\gamma_e$ of element $e$,
\begin{align}
&\nonumber\int_{\gamma}\pd{G(\vec{\eta}';\vec{\eta})}{\vec{n}}\phi(\vec{\eta}) r \,\mathrm{d}\gamma(\vec{\eta})
\approx\sum_e \sum_k p_{e_k}
\int_{\gamma_e}\pd{G(\vec{\eta}';\vec{\eta})}{\vec{n}}\mathcal{N}^{(e)}_k(\xi) r \,\mathrm{d}\gamma(\vec{\eta})\\
=&\nonumber\,\sum_e \sum_k p_{e_k}
\int_0^1\mathcal{N}^{(e)}_k(\xi)
\frac{J}{\pi \sqrt{a+b}}\bigg\{
\left[\frac{n_r}{2}+\frac{\vec{n}\cdot(\vec{x}'-\vec{x})}{\|\vec{x}'-\vec{x}\|^2}r\right]E(m)
-\frac{n_r}{2}K(m)\bigg\}
\,\mathrm{d}t\\
=&\,\sum_e \sum_k p_{e_k}
\int_0^1
\frac{\mathcal{N}^{(e)}_k(\xi)}{\pi \sqrt{a+b}}\bigg\{
\left[\frac{\dot{r}(z'-z)-\dot{z}(r'-r)}{(r'-r)^2+(z'-z)^2}r
-\frac{\dot{z}}{2}\right]E(m)
+\frac{\dot{z}}{2}K(m)\bigg\}
\,\mathrm{d}t
\end{align}
Technically we don't need to treat integrand involving elliptic integral of the second kind $E(m)$.
However, convergence rate of standard Gauss-Legendre quadrature  depends magnitude of derivatives.
\begin{equation}
f_E^\mathrm{double}=\frac{1}{\pi \sqrt{a+b}}
\left[\frac{\dot{r}(z'-z)-\dot{z}(r'-r)}{a-b}r
-\frac{\dot{z}}{2}\right],\quad
f_K^\mathrm{double}=\frac{1}{\pi \sqrt{a+b}}
\frac{\dot{z}}{2}
\end{equation}
If we define integrand $I^{(o)}$ for source location $\vec{\eta}'$ outside of $\gamma_e$
and $I^{(i)}$ for $\vec{\eta}'$ interior to $\gamma_e$,
\begin{align}
I^{(o)}(t) &= \left[ f_E^\mathrm{double} E(m)+f_K^\mathrm{double} K(m)\right]\mathcal{N}^{(e)}_k(\xi)\\
I^{(i)}(t,\tau) &= \left[f_E^\mathrm{double}R_E(m,\tau)+ f_K^\mathrm{double}R_K(m,\tau)\right]\mathcal{N}^{(e)}_k(\xi)\\
I^{(i)}_\mathrm{log}(t) &=-2 f_E^\mathrm{double}Q_E(m) -2 f_K^\mathrm{double}Q_K(m) 
\end{align}
When $\vec{\eta'}$ is located the symmetry axis where $r'=0$,
we can simplify the elliptic integrals,
\begin{align}
\int_{\gamma}G(\vec{\eta}';\vec{\eta})
\pd{\phi}{\vec{n}}(\vec{\eta}) r \,\mathrm{d}\gamma(\vec{\eta})
&\approx\sum_e \sum_k q_{e_k}
\int_0^1\mathcal{N}^{(e)}_k(\xi)r\frac{   J }{
2 \sqrt{r+(z-z')^2}}\,\mathrm{d}t\\
\int_{\gamma}\pd{G(\vec{\eta}';\vec{\eta})}{\vec{n}}\phi(\vec{\eta}) r \,\mathrm{d}\gamma(\vec{\eta})
&\approx\sum_e \sum_k p_{e_k}
\int_0^1\mathcal{N}^{(e)}_k(\xi)r \frac{  
\dot{z}r+\dot{r}(z-z')
}{
2\big[r^2+(z'-z)^2\big]^{3/2}
}\,\mathrm{d}t
\end{align}
<<<<<<< HEAD
Standard Gauss-Legendre quadrature is sufficient
to evaluate the above integral since
the integrands have well-defined limit as $\vec{\eta}\to\vec{\eta}'$
assuming $\ddot{z}/\dot{r}$ is finite at $r=0$.


\subsection{Matrix assembly}
\begin{table}
  \begin{center}
\def~{\hphantom{0}}
  \begin{tabular}{rl|l}
Element order & $o$\\
Number of elements & $n$\\
Number of nodes & $ n \times o +1$\\
Nodal index of source point & $ i$& $ 0\le i \le  n \times o$\\
Elemental index of receiving point & $ j$& $ 0\le j \le  n-1$\\
Local node index & $k$ & $0\le k \le o$\\
Indices of elements $\ni$ $i$-th node if $i \textrm{ mod }  o =0$
& $ \lfloor i / o\rfloor-1$, $ \lfloor i / o\rfloor$& \\
Indices of elements $\ni$  $i$-th node  if $i \textrm{ mod }  o \neq 0$
& $ \lfloor i / o\rfloor-1$&\\
=======
The above integrands have well-defined limit as $\vec{\eta}\to\vec{\eta}'$
assuming $\ddot{z}/\dot{r}$ is finite at $r=0$.

\subsection{Assembly}
\begin{table}
  \begin{center}
\def~{\hphantom{0}}
  \begin{tabular}{rl|r}
Element order & $o$\\
Number of elements & $n$\\
Number of nodes & $ n \times o +1$\\
Nodal index of source point & $ i$\\
Elemental index of receiving point & $ j$\\
Local node index & $k$ & $0\le k \le o$\\
Indices of elements $\ni$ $i$-th node if $i \textrm{ mod }  o =0$
& $ \lfloor i / o\rfloor-1$, $ \lfloor i / o\rfloor$& check $ 0\le$ id $\le n-1$\\
Indices of elements $\ni$  $i$-th node  if $i \textrm{ mod }  o \neq 0$
& $ \lfloor i / o\rfloor-1$& check $ 0\le$ id $\le n-1$\\
>>>>>>> 261252fba4bf040dc7cc35b34ba8820bbadbae79
Nodal indices of $j$-th element  & $j \times o +  k$ & $0\le k \le o$\\
  \end{tabular}
  \caption{Index involved}
  \label{tab:kd}
  \end{center}
\end{table}

\begin{algorithm}
  \caption{Counting mismatches between two packed strings
    \label{alg:packed-dna-hamming}}
  \begin{algorithmic}[1]
%\algloop{For}
    \Function{Distance}{$x$, $e$}
    \For{$ 0\le i \le N_x - 1$ } \hfill\text{\# We can parallelize this loop}
    \If{$i \textrm{ mod } o = 0$}     
\State    $I_\mathrm{lower}$  =    $I_\mathrm{upper}$  
          \Else
\State              $I_\mathrm{lower} = \lfloor i/o\rfloor - 1$,\quad
              $I_\mathrm{upper} = \lfloor i/o\rfloor$  
          \EndIf
    \For{$ 0\le i \le N_x - 1$ } 
    
    \EndFor
    
    
    \EndFor
    \EndFunction
  \end{algorithmic}
\end{algorithm}
<<<<<<< HEAD
The collocation procedure for a single $C^1$ continuous boundary $\gamma$ leads
to a matrix equation,
=======

>>>>>>> 261252fba4bf040dc7cc35b34ba8820bbadbae79
\begin{equation}
(\mat{B}+\mat{D})\vec{p} =\mat{H}\vec{p} = \mat{S}\vec{q}
\end{equation}
Consider two adjacent boundaries $\gamma_0$ and $\gamma_1$.
<<<<<<< HEAD
The assembly process can be decomposed into
four parts: 
$\{\vec{\eta'}\in\gamma_0,\vec{\eta'}\in\gamma_1\}\otimes\{\vec{\eta}\in\gamma_0,\vec{\eta}\in\gamma_1\}$
which leads to a block matrix equation,
=======
We have a block matrix equation,
>>>>>>> 261252fba4bf040dc7cc35b34ba8820bbadbae79
\begin{equation}
\begin{bmatrix}
\mat{H}_{00}&\mat{H}_{01}\\\mat{H}_{10}&\mat{H}_{11}
\end{bmatrix}
\begin{bmatrix}
\vec{p}_0\\\vec{p}_1
\end{bmatrix}
=\begin{bmatrix}
\mat{S}_{00}&\mat{S}_{01}\\\mat{S}_{10}&\mat{S}_{11}
\end{bmatrix}
\begin{bmatrix}
\vec{q}_0\\\vec{q}_1
\end{bmatrix}
\end{equation}
We have two identical rows in $\mat{H}$ and $\mat{S}$ 
and extra equation for continuity of potential,
\begin{equation}\left.\begin{aligned}
\mathrm{row}_{-1}(\mat{H}_{00})\cdot\vec{p}_0
+\mathrm{row}_{-1}(\mat{H}_{01})\cdot\vec{p}_1
&=\mathrm{row}_{-1}(\mat{S}_{00})\cdot\vec{q}_0
+\mathrm{row}_{-1}(\mat{S}_{01})\cdot\vec{q}_1\\
\mathrm{row}_{0}(\mat{H}_{10})\cdot\vec{p}_0
+\mathrm{row}_{0}(\mat{H}_{11})\cdot\vec{p}_1
&=\mathrm{row}_{0}(\mat{S}_{10})\cdot\vec{q}_0
+\mathrm{row}_{0}(\mat{S}_{11})\cdot\vec{q}_1\\
(p_0)_{-1}-(p_1)_{0}&=0
\end{aligned}\right\}\end{equation}
We simply replace one of identical rows with the continuity condition,
\begin{equation}\begin{aligned}
\mathrm{row}_{-1}(\mat{H}_{00}) &= [0, \,\cdots ,0 , 1],&
\mathrm{row}_{-1}(\mat{H}_{01}) &= [-1,0, \,\cdots ,0 ]\\
\mathrm{row}_{-1}(\mat{S}_{00}) &= [0, \,\cdots ,0 , 0],&
\mathrm{row}_{-1}(\mat{S}_{01}) &= [0,0, \,\cdots ,0 ]
\end{aligned}\end{equation}
For mixed-type boundary condition, i.e., Dirichlet and Neumann on different segments of the boundary,
we simply rearrange the block matrices,
\begin{equation}
\begin{bmatrix}
\mat{H}_{00}&-\mat{S}_{01}\\\mat{H}_{10}&-\mat{S}_{11}
\end{bmatrix}
\begin{bmatrix}
\vec{p}_0\\\vec{q}_1
\end{bmatrix}
=\begin{bmatrix}
\mat{S}_{00}&-\mat{H}_{01}\\\mat{S}_{10}&-\mat{H}_{11}
\end{bmatrix}
\begin{bmatrix}
\vec{q}_0\\\vec{p}_1
\end{bmatrix}
\end{equation}
Note the continuity between discrete potential vectors $\vec{p}_0$ and $\vec{p}_1$
<<<<<<< HEAD
is still implied. What cannot be achieved is to prescribe identical value of potential both before and after the transition point.
In this case, continuity of potential can no longer provide an extra equation.
=======
is still implied.
>>>>>>> 261252fba4bf040dc7cc35b34ba8820bbadbae79

\subsection{Benchmark}
We validate our implementation of boundary integral solver for test problems posed 
on a smooth boundary $\gamma$ and on a piecewise smooth boundary $\gamma_0\cup\gamma_1$.
The latter contains a geometric discontinuity of a 90$^\circ$ corner
at the transition point between $\gamma_0$ and $\gamma_1$.
Consider the following parametrisation of the boundaries,
\begin{align}
\gamma_0&=\left\{
2 \Big(1+\frac{1}{4}\cos(8\theta - \pi)\Big)
\begin{pmatrix}
\cos\theta\\
\sin\theta
\end{pmatrix}
\Bigm|  \theta \in [0,\pi/2]\right\}\\
\gamma_1&=
\left\{
\begin{pmatrix}
t\\0
\end{pmatrix}
\Bigm|  \theta \in [0,3/2]\right\}
\end{align}
The general form of axisymmetric harmonic potential can be constructed from Legendre polynomial $P_\ell$ of order $\ell$.
We consider the interior problem for the potential $\phi$,
\begin{equation}\label{eq:test_parametrisation}
\phi = (r^2+z^2)P_2\Big(\frac{z}{\sqrt{r^2+z^2}}\Big),\quad
\pd{\phi}{\vec{n}} = \vec{n}\cdot (-r , 2 z )
\end{equation}
If we prescribe the potential value of $\phi$ on $\gamma_0$ and 
its normal flux $\partial \phi /\partial \vec{n}$ on $\gamma_1$,
our solver is expected to reproduce $\partial \phi /\partial \vec{n}$ on $\gamma_0$
and $\phi$ on $\gamma_1$,
\begin{equation}\label{eq:test_problem}
\textrm{Test problem:}\quad\textrm{given}\left\{
\begin{aligned}
\phi &\textrm{ on }\gamma_0\\
\partial \phi /\partial \vec{n}&\textrm{ on }\gamma_1
\end{aligned}\right.,
\,\,\textrm{find}\left\{
\begin{aligned}
\partial \phi /\partial \vec{n}&\textrm{ on }\gamma_0\\
\phi &\textrm{ on }\gamma_1
\end{aligned}\right.
\end{equation}
In addition for a plane curve given parametrically as $\gamma(t) = (r(t), z(t))$,
the total curvature $2\kappa$ of the surface of revolution obtained by rotating curve $\gamma$
about the $z$-axis is given by,
\begin{equation}
2\kappa= \frac{\dot{r}\ddot{z} - \dot{z}\ddot{r}}{(\dot{r}^2+\dot{z}^2)^{\frac{3}{2}}}
+\frac{1}{r}\frac{\dot{z}}{\sqrt{\dot{r}^2+\dot{z}^2}}
\end{equation}
We verify the accuracy of quintic spline interpolation against the analytic form derived from
the parametrisation (\ref{eq:test_parametrisation}). All errors between numerical and analytical solutions
are measured in the $l^\infty$-norm,
\begin{align}
\|\textrm{err}_\mathrm{d}\|_\infty &= 
\max_{\vec{x}_i\in \gamma_1}{\lvert p^\mathrm{num}_i-\phi(\vec{x}_i)\rvert}\label{eq:errorDirichlet}\\
\|\textrm{err}_\mathrm{n}\|_\infty &= 
\max_{\vec{x}_i\in \gamma_0}{\lvert q^\mathrm{num}_i-\partial\phi/\partial\vec{n}(\vec{x}_i)\rvert}\label{eq:errorNeumann}\\
\|\textrm{err}_\mathrm{c}\|_\infty &= 
\max_{\vec{x}_i\in \gamma_0}{\lvert 2\kappa^\mathrm{spline}_i-2\kappa(\vec{x}_i)\rvert}\label{eq:errorCurvature}
\end{align}
In figure \ref{fig:test_result}(a) and \ref{fig:test_result}(b) 
we plot these errors against the total number of degree of freedom (DOF) used by the solver.
\begin{figure}
  \centering
  \includegraphics[height = 5.5 cm]{fig01.pdf}% Images in 100% size
  \caption{
Error convergence  between numerical and analytic solutions measured in $l^\infty$-norm against degrees of freedom (DOF) used for the test problem (\ref{eq:test_problem}):
(a) Boundaries $\gamma_0$ (solid) and $\gamma_1$ (dashed) given 
by parametrisation (\ref{eq:test_parametrisation}).
(b) Error convergence: curvature error (\ref{eq:errorCurvature}) ($\blacktriangledown$) on $\gamma_0$;
Neumann error (\ref{eq:errorNeumann}) on $\gamma_0$ 
with linear ($\CIRCLE $) and quadratic ($\blacksquare$) shape functions;
Dirichlet error (\ref{eq:errorDirichlet})  on $\gamma_1$ 
with linear ($\ocircle $) and quadratic ($\Box$) shape functions.
Inset: linear fit (dashed line) of the last five points of each error.
  }
\label{fig:test_result}
\end{figure}

\section{Self-similar Taylor cone}
<<<<<<< HEAD
\subsection{Preparation for iteration}
=======
\subsection{Prepare for iteration}
>>>>>>> 261252fba4bf040dc7cc35b34ba8820bbadbae79
In the last section we outline the numerical procedure to solve Laplace equation subject to mixed boundary conditions.
Initial guess is a $C^3$ continuous function $f_\mathrm{guess}$,
\begin{equation}
f_\mathrm{guess}(r)=\left\{\begin{aligned}
&f_0r+f_2r^2+f_4r^4+f_6r^6,&&\textrm{if }r\le r_c\\
&c_0 r + \frac{c_1}{\sqrt{r}} +\frac{c_3}{r^{7/2}}+\frac{c_4}{r^{5}},&&\textrm{if }r>r_c
\end{aligned}\right.
\end{equation}
where $r_c$ is the connection point where continuities up to the third derivative are enforced.
Given a reasonable $c_1$, function $f_\mathrm{guess}(r)$ agrees with analytic prediction in the far-field.
The effect of varying $r_c$ and $c_1$ is illustrated in figure \ref{fig:initialGuess}.
\begin{figure}
  \centering
  \includegraphics[height =  4.65cm]{fig02.pdf}% Images in 100% size
  \caption{$C^3$ function $f_\mathrm{guess}(r)$: increasing $r_c$ (right) and $c_1$ (left)  }
\label{fig:initialGuess}
\end{figure}


We clamp the spline function, its tangent direction and total curvature at truncation point $r_*$
\begin{equation}
\dot{z}=\dot{r}\dd{f}{r},\quad
\ddot{z}
=\frac{\ddot{r} \dot{z}+2\kappa   (\dot{r}^2+\dot{z}^2)^{3/2}
 }{\dot{r} }
-
\frac{\dot{z} \left(\dot{r}^2+\dot{z}^2\right)}{\dot{r} r}
\end{equation}
Forward, central and backward finite difference approximation to first derivative 
with three data points $\{(x_i,f_i)\}_{i=0,1,2}$,
\begin{align}
f'(x_0)&\approx-f_0 \left(\frac{1}{h_0+h_1}+\frac{1}{h_0}\right)+f_1 \left(\frac{1}{h_0}
+\frac{1}{h_1}\right)-f_2 \left(\frac{1}{h_1}-\frac{1}{h_0+h_1}\right)\\
f'(x_1)&\approx+f_0 \left(\frac{1}{h_0+h_1}-\frac{1}{h_0}\right)+f_1 \left(\frac{1}{h_0}
-\frac{1}{h_1}\right)+f_2 \left(\frac{1}{h_1}-\frac{1}{h_0+h_1}\right)\\
f'(x_2)&\approx-f_0 \left(\frac{1}{h_0+h_1}-\frac{1}{h_0}\right)-f_1 \left(\frac{1}{h_0}
+\frac{1}{h_1}\right)+f_2 \left(\frac{1}{h_0+h_1}+\frac{1}{h_1}\right)\\
f''(x_0)&\approx f''(x_1)\approx f''(x_2)\approx\frac{2}{h_0(h_0+h_1)} f_ 0-\frac{2 }{h_0 h_1}f_ 1
+ \frac{2}{ h_ 1(h_ 0+h_ 1)}f_ 2
\end{align}
where $h_i = x_{i+1} - x_i$.
For curvature purpose, we only list the forward and
backward finite difference approximation to the second derivative with four-point stencil,
\begin{align}
\nonumber f''(x_3)\approx&&-\frac{2 (h_1+2 h_2)}{h_0 (h_0+h_1) (h_0+h_1+h_2)} f_0
&&+\frac{2  (h_0+h_1+2 h_2)}{h_0 h_1 (h_1+h_2)}f_1\\
\phantom{\approx}&&-\frac{2  (h_0+2 (h_1+h_2))}{h_1 h_2 (h_0+h_1)}f_2
&&+\frac{2  (h_0+2 h_1+3 h_2)}{h_2 (h_1+h_2) (h_0+h_1+h_2)}f_3
\end{align}
Thus we use backward finite difference method to approximate the first and second derivatives of $r$-component of the spline at the truncation point $r_*$.
With prescribed slope and curvature, we can compute the first and second derivatives of $z$-component.

<<<<<<< HEAD
Gradient of axisymmetric harmonic function,
=======
Axisymmetric harmonic function,
>>>>>>> 261252fba4bf040dc7cc35b34ba8820bbadbae79
\begin{align}
\pd{}{r}R^\ell P_\ell(\cos\theta)
&=\frac{ \ell r^2-(\ell +1)z^2 }{r}R^{\ell-2}P_\ell\left(\frac{z}{R}\right)
+\frac{(\ell+1) z }{r} R^{\ell-1}P_{\ell+1}\left(\frac{z}{R}\right)\\
\pd{}{z}R^\ell P_\ell(\cos\theta)
&=(2 \ell+1) z R^{\ell-2} P_\ell\left(\frac{z}{R}\right)-(\ell+1) R^{\ell-1} P_{\ell+1}\left(\frac{z}{R}\right)\\
\pd{}{r}R^{-\ell-1} P_\ell(\cos\theta)&=\frac{(\ell+1) z }{r}R^{-\ell-2} P_{\ell+1}\left(\frac{z}{R}\right)
-\frac{\ell+1 }{r}R^{-\ell-1} P_\ell\left(\frac{z}{R}\right)\\
\pd{}{z}R^{-\ell-1} P_\ell(\cos\theta)&=-(\ell+1) R^{-\ell-2} P_{\ell+1}\left(\frac{z}{R}\right)
\end{align}
Note $\partial /\partial r$ of these axisymmetric harmonic functions at $z=0$ is well-defined and always zero due to symmetry.


<<<<<<< HEAD
\subsection{Newton-Raphson iteration}
We look for possible root to the self-similar Bernoulli's equation,
\begin{equation}
\mathcal{B}(\vec{\eta})=\frac{2}{3}\vec{\eta}\cdot \nabla \phi  -\frac{1}{3}\phi
+\frac{1}{2}|\nabla \phi|^2-2\kappa-\frac{1}{2}|\nabla \psi|^2
\end{equation}
subject to the boundary conditions,
\begin{equation}
\frac{2}{3}\vec{n}\cdot \vec{\eta}+\pd{\phi}{\vec{n}}=0,\quad
\psi =0 \quad\textrm{on}\quad\gamma_0
\end{equation}
We rewrite $\mathcal{B}$ in terms of position $\vec{\eta}$ and velocity potential $\phi$ on $\gamma_0$,
=======
\subsection{Newton iteration}
We aim to find the root 
\begin{equation}
\mathcal{B}(\vec{\eta})=\frac{2}{3}\vec{\eta}\cdot \nabla \phi  -\frac{1}{3}\phi+\frac{1}{2}|\nabla \phi|^2-2\kappa
\end{equation}
with 
\begin{equation}
\frac{2}{3}\vec{n}\cdot \vec{\eta}+\pd{\phi}{\vec{n}}=0
\end{equation}
Rewrite $\mathcal{B}$,
>>>>>>> 261252fba4bf040dc7cc35b34ba8820bbadbae79
\begin{align}
\nonumber \mathcal{B}(\vec{\eta})&=\frac{2}{3}(\vec{n}\cdot\vec{\eta}) \pd{\phi}{\vec{n}}+
\frac{2}{3}(\vec{s}\cdot\vec{\eta}) \pd{\phi}{\vec{s}}
  -\frac{1}{3}\phi+\frac{1}{2}\left( \pd{\phi}{\vec{n}}\right)^2
  +\frac{1}{2}\left( \pd{\phi}{\vec{s}}\right)^2-2\kappa\\
  &=\frac{1}{2}\left( \pd{\phi}{\vec{s}}\right)^2+
  \frac{2}{3}(\vec{s}\cdot\vec{\eta}) \pd{\phi}{\vec{s}}
  -\frac{1}{3}\phi-\frac{2}{9}\left( \vec{n}\cdot \vec{\eta}\right)^2
<<<<<<< HEAD
  -2\kappa-\frac{1}{2}\left( \pd{\psi}{\vec{n}}\right)^2=0
\end{align}
This problem is nonlinear in velocity field which is normal for inertia-dominate flow
but more importantly nonlinear in the geometry.
\begin{algorithm}
  \caption{Nonlinear iteration \label{alg:taylorConeNewton}}
=======
  -2\kappa
\end{align}

\begin{algorithm}
  \caption{Nonlinear iteration \label{alg:??}}
>>>>>>> 261252fba4bf040dc7cc35b34ba8820bbadbae79
  \begin{algorithmic}[1]
%\algloop{For}
    \Function{Newton iteration}{$c_1$, $a_0$, $r_*$} \Comment{truncation distance $r_*$}
    \State compute $c_0,\dots , c_4$ \Comment{$c_2=0$}
    \State initialize guess shape $\gamma_0$\Comment{fix 1st and 2nd derivatives at the end}
<<<<<<< HEAD
    \State initialize fluid and vacuum patches, $\gamma_1$ and $\gamma_2$
    \While{$\textrm{error}(\vec{\mathcal{B}})<\textrm{tolerance}$} \Comment{$l^\infty$ error norm}
    \ForAll{knots $\vec{\eta}_i$ of spline $\gamma_0$} \Comment{exclude ending knot}
=======
    \State initialize patch $\gamma_1$
    \While{$\textrm{error}(\vec{\mathcal{B}})<\textrm{tolerance}$} \Comment{ $l^\infty$ error norm}
    \ForAll{knots $\vec{\eta}_i$ of spline $\gamma_0$} 
>>>>>>> 261252fba4bf040dc7cc35b34ba8820bbadbae79
    \State perturb $\vec{\eta}_i \to \vec{\eta}_i+\epsilon \vec{n}_i$
    \State compute ending derivatives $\dot{r},\ddot{r},\dot{z},\ddot{z}$
    \State initialize perturbed spline $\underline{\gamma}_0$    
     \State initialize perturbed Neumann condition on $\underline{\gamma}_0$    
    \State compute solution $\underline{\phi}$ on $\underline{\gamma}_0$    
<<<<<<< HEAD
     \State compute perturbed residue vector $\underline{\vec{\mathcal{B}}}$ \Comment{for knots only}
     \State $\mathrm{col}_i(\mat{J}) = (\underline{\vec{\mathcal{B}}}-\vec{\mathcal{B}})/\epsilon$
    \EndFor
    \State solve $\mat{J}\, \delta\vec{n}=-\vec{\mathcal{B}}$ 
				\Comment{$\mathrm{dim}(\mat{J}) = \#_\mathrm{node}\times( \#_\mathrm{knot} - 1) $}
    \State update $\vec{\eta}_i \to \vec{\eta}_i+\alpha \delta\vec{n}_i$ with damping rate $\alpha$
    \State compute new residue vector $\vec{\mathcal{B}}$
    \EndWhile        
=======
     \State compute perturbed residue vector $\underline{\vec{\mathcal{B}}}$ on knots only
     \State $\mathrm{col}_i(\mat{J}) = (\underline{\vec{\mathcal{B}}}-\vec{\mathcal{B}})/\epsilon$
    \EndFor
    \State solve $\mat{J}\, \delta\vec{n}=-\vec{\mathcal{B}}$
    \State update $\vec{\eta}_i \to \vec{\eta}_i+\alpha \delta\vec{n}_i$ with damping rate $\alpha$
    \State compute new residue vector $\vec{\mathcal{B}}$
    \EndWhile        
%    \For{$ 0\le i \le N_x - 1$ } \hfill\text{\# We can parallelize this loop}
%    \If{$i \textrm{ mod } o = 0$}     
%\State    $I_\mathrm{lower}$  =    $I_\mathrm{upper}$  
%          \Else
%\State              $I_\mathrm{lower} = \lfloor i/o\rfloor - 1$,\quad
%              $I_\mathrm{upper} = \lfloor i/o\rfloor$  
%          \EndIf
%    \For{$ 0\le i \le N_x - 1$ } 
%    
%    \EndFor
%    
%    
%    \EndFor
>>>>>>> 261252fba4bf040dc7cc35b34ba8820bbadbae79
    \EndFunction
  \end{algorithmic}
\end{algorithm}


<<<<<<< HEAD
\section{Results and discussion}
\subsection{Burton \& Taborek}
We first recover the self-similar cone formation at the end of an inviscid drop reported by
\cite{Burton11}.
\begin{figure}  
% \includegraphics[height = 5.5 cm]{fig03BT.pdf}% Images in 100% size
  \includegraphics[height = 4.7 cm]{fig03_a_light.pdf}\,
  \includegraphics[height = 4.7 cm]{fig03BT_light.pdf}\\
  \includegraphics[height = 4.7 cm]{fig03_b_light.pdf}\,
  \includegraphics[height = 4.7 cm]{fig03BT_light.pdf}
  \caption{\cite{Burton11}
??\label{fig:??}
}
\end{figure}
=======
>>>>>>> 261252fba4bf040dc7cc35b34ba8820bbadbae79













\bibliographystyle{jfm}
% Note the spaces between the initials
<<<<<<< HEAD
\bibliography{taylorConeNumeric}
=======
\bibliography{taylorCone}
>>>>>>> 261252fba4bf040dc7cc35b34ba8820bbadbae79

\end{document}
